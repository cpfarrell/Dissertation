\documentclass [PhD] {uclathes}



% \input {mymacros}                         % personal LaTeX macros
\usepackage{amsmath,amssymb}% for advanced math typesetting
\usepackage{hyperref}
\usepackage{graphicx}
\usepackage{multicol}
\usepackage{multirow}
\usepackage{oldlfont}
%\usepackage{natbib}
\usepackage{longtable}
\DeclareMathOperator{\sgn}{sgn}% Declare the sgn operator 
\DeclareMathOperator{\sign}{sign}% for those who insist on sign instead
\setcounter{tocdepth}{1}

%%%%%%%%%%%%%%%%%%%%%%%%%%%%%%%%%%%%%%%%%%%%%%%%%%%%%%%%%%%%%%%%%%%%%%
%
% Usually things live in separate flies.
%
% \input {prelim}                           % preliminary page info

%%%%%%%%%%%%%%%%%%%%%%%%%%%%%%%%%%%%%%%%%%%%%%%%%%%%%%%%%%%%%%%%%%%%%%%%
%                                                                      %
%                          PRELIMINARY PAGES                           %
%                                                                      %
%%%%%%%%%%%%%%%%%%%%%%%%%%%%%%%%%%%%%%%%%%%%%%%%%%%%%%%%%%%%%%%%%%%%%%%%

\title          {Optimization of muon timing and searches for heavy
                long-lived charged particles with the
                Compact Muon Solenoid detector at the
                Large Hadron Collider}
\author         {Christopher Patrick Farrell}
\department     {Physics and Astronomy}
% Note:  degreeyear should be optional, but as of  5-Feb-96
% it seems required or you get a year of ``2''.   -johnh
\degreeyear     {2013}

%%%%%%%%%%%%%%%%%%%%%%%%%%%%%%%%%%%%%%%%%%%%%%%%%%%%%%%%%%%%%%%%%%%%%%%%

\chair          {Jay Hauser}
\member         {David Saltzberg}
\member         {Michael Gutperle}
\member         {Peter Felker}

%%%%%%%%%%%%%%%%%%%%%%%%%%%%%%%%%%%%%%%%%%%%%%%%%%%%%%%%%%%%%%%%%%%%%%%%

\dedication     {\textsl{To my parents who have always been supportive of me}}

%%%%%%%%%%%%%%%%%%%%%%%%%%%%%%%%%%%%%%%%%%%%%%%%%%%%%%%%%%%%%%%%%%%%%%%%

\acknowledgments {
I want to say thank you to my advisor Jay Hauser whose teaching and advice allowed me to grow into a contributing member of the scientific community.
%for showing me how to investigate physics in a thorough and detailed manner.
Jay has always been available to aid in my understanding of problems and to give ideas when I was unsure of what to do going forward.
I could not have completed this work without him.

My development as a physicist has been shaped by discussions and guidance by my colleagues on CMS. 
The list includes but is not limited to 
Greg Rakness, Amanda Deisher, Mikhael Ignatenko, Chad Jarvis, Loic Quertenmont, and Todd Adams.

I am also grateful to David Saltzberg, Michael Gutperle, and Peter Felker for being willing to be on my committee and review my dissertation.

Finally I want to thank my family for their constant support. Especially, I want to thank my parents who have always worked to make my life as happy and fulfilling as possible.
}

%%%%%%%%%%%%%%%%%%%%%%%%%%%%%%%%%%%%%%%%%%%%%%%%%%%%%%%%%%%%%%%%%%%%%%%%

\vitaitem{1985}{Born, New Jersey, USA.}
\vitaitem{2008}{B.S.~Physics, University of Florida.}
\vitaitem{2008--present}{Research Assistant, University of California, Los Angeles.}

%%%%%%%%%%%%%%%%%%%%%%%%%%%%%%%%%%%%%%%%%%%%%%%%%%%%%%%%%%%%%%%%%%%%%%%%

\publication{
S.~Chatrchyan {\it et~al.} [CMS Collaboration],
``Search for heavy long-lived charged particles in $pp$ collisions at
  $\sqrt{s}=7$ TeV.''
{\em Phys.Lett.}, B713:408--433, 2012.
}

\publication{
The CMS Collaboration.
``Performance of CMS muon reconstruction in pp collision events at
  $\sqrt{s}=7$ TeV.''
{\em Journal of Instrumentation}, 7:2P, October 2012.
}

%%%%%%%%%%%%%%%%%%%%%%%%%%%%%%%%%%%%%%%%%%%%%%%%%%%%%%%%%%%%%%%%%%%%%%%%

\newcommand{\rad}{rad}
\newcommand{\ias}{$I_{as}$}
\newcommand{\iasp}{$I_{as}^\prime$}
%\newcommand{\id}{$I_{d}$}                                                                                                                                                         
\newcommand{\ih}{$I_{h}$}
\newcommand{\dedx}{$dE/dx$}
\newcommand{\tof}{TOF}
\newcommand{\pt}{$p_{T}$}
\newcommand{\invbeta}{$\beta^{-1}$}
\newcommand{\invbetac}{$1/\beta, $}
\newcommand{\invbetap}{$1/\beta. $}
\newcommand{\tkonly}{{\em track only}}
\newcommand{\tktof}{{\em muon+track}}
\newcommand{\multi}{{\em multiple charge}}
\newcommand{\muononly}{{\em muon only}}
\newcommand{\fract}{{\em fractional charge}}

\newcommand{\stau}{$\tilde \tau _1 \ $}
\newcommand{\smu}{$\tilde \mu _1 \ $}
\newcommand{\sel}{$\tilde e _1 \ $}
\newcommand{\stp}{$\tilde t _1 \ $}
\newcommand{\gluino}{$\tilde g \ $}
\newcommand{\ETM}{$E_t^{miss} \ $}

%%%%%%%%%%%%%%%%%%%%%%%%%%%%%%%%%%%%%%%%%%%%%%%%%%%%%%%%%%%%%%%%%%%%%%%%

\abstract       {
Proton--proton collisions at the Large Hadron Collider at $\sqrt{s} =$ 7 and 8 TeV are studied using the Compact Muon Solenoid (CMS) detector.
The measurement and improvements to the arrival time of particles to the muon system of CMS is detailed.
The timing is used to associate the particle with the correct proton--proton crossing and to classify the particle.
%One type of particle timing can help identify is new long-lived charged particles predicted in many theories of new physics, such as supersymmetry.
Additionally, four analyses are presented that use timing and ionization energy loss to search for the production of long-lived charged particles 
predicted in many theories of new physics. The searches are sensitive to a variety of signatures,
including the possibility that the particles will only be detectable during part of their passage through the CMS detector.
Limits are placed on the production of long-lived gluinos, stops, staus, and multiply charged particles.
The limits are the most stringent in the world to date.
}

%%%%%%%%%%%%%%%%%%%%%%%%%%%%%%%%%%%%%%%%%%%%%%%%%%%%%%%%%%%%%%%%%%%%%%%%


\begin {document}
%\bibliographystyle{plain}
\makeintropages

%%%%%%%%%%%%%%%%%%%%%%%%%%%%%%%%%%%%%%%%%%%%%%%%%%%%%%%%%%%%%%%%%%%%%%
%
% Ordinarily each chapter (at least) is in a separate file.
\chapter{Introduction}

%Throughout human history there has been a desire to understand nature at its most fundamental level.
%Particle physics is concerned with studying the nature of fundamental particles and the interactions between them. 
Particle physics is concerned with studying what matter is made of and how it interacts at a fundamental level.
%he matter could be everyday objects or far away stars.
The desire to understand nature at its most basic level has long been of interest to humankind, dating back to at least the ancient Greek philosophers.
The field of particle physics began to come into its own beginning with the discovery of the electron in 1897 by J.J. Thompson~\cite{griffiths2008introduction}.
In the more than a century since that discovery, the field has been revolutionized many times. Whenever it seemed that the final piece of the puzzle was within reach,
a surprising result would be found: fundamental particles shown to be composites, inviolate symmetries found to be broken,
or an ever-expanding universe.

The current state of knowledge in particle physics is contained in the standard model (SM) of particle physics. The SM contains all of the known
particles and the way that they interact with one another. The SM has been developed over the last forty years and has proven to be a very successful theory. 
Numerous experiments have validated the theory with the discovery of predicted particles or agreement of parameter values.

The last particle to be found that is predicted by the SM is the Higgs Boson.
% predicted in 1974 (cite Higgs theory).
The discovery of the Higgs Boson is one of the reasons the Large Hadron Collider (LHC) was built outside of Geneva, Switzerland. 
The LHC collides protons at an energy higher than any previous experiment at an extremely high rate~\cite{1748-0221-3-08-S08001}
producing many particles in the collisions.
% including Higgs Bosons if they exist. 
Surrounding the points where the LHC brings the protons to a collision
are experiments designed to measure and reconstruct the particles emanating from the collisions. Two of these experiments, CMS and
ATLAS, announced in June of 2012 the discovery of a new particle with properties like that of the SM Higgs Boson~\cite{Chatrchyan:2013lba, Aad:2012tfa}.
%, possibly indicating that the last piece of the SM has been found.

%However it may be that the discovered particle is not the SM Higgs but comes from another source (cite papers saying not Higgs).
Throughout the history of particle physics, just when the theory looks to be the most stable can be when a discovery revolutionizes the field.
There are many models which predict physics beyond the SM~\cite{Martin:1997ns, Tata:1997uf} 
that would be produced by the LHC and could be detected by the LHC experiments.
One of the most popular models is supersymmetry (SUSY) where all of the SM particles are given a superpartner with spin different by one half.
Some of these models predict the production of new heavy (meta-)stable charged particles (HSCP) 
which would directly interact with the LHC experiments. All of the long-lived SM particles produced by the LHC have a small mass, meaning that a discovery of an HSCP
would be a clear indication that a new theory must be developed to explain these particles.

There are numerous different types of HSCP predicted in the models of new physics, and the different types can leave exotic signatures in the LHC detectors.
Some types of HSCPs would form composite objects with SM particles and undergo nuclear interactions with the detector material
that change the SM particles in the composites. This could lead to the composite changing its electric charge during its propagation through the detector.
Other HSCP could be produced with electric charge not equal to $e$, the charge of an electron, unlike almost all electrically charged particles 
expected to be produced at the LHC.

Multiple complementary searches for HSCP produced at the LHC were carried out using data collected by the CMS experiment and are presented in this work.
Each of the searches is designed to be sensitive to different signatures that HSCP could have in CMS but many of the tools and techniques used are
common between searches. The searches exploit the experimental signatures of HSCP that allow them to be separated from the very large background of SM particles.
Requirements are placed on these characteristics and the residual background due to SM particles is evaluated. The data are then checked to see whether
the observation is consistent with this background. If it is not consistent, then this indicates the presence of a new particle beyond the SM while if it
is consistent then this places limits on models of physics beyond the SM.

HSCP produced at the LHC are expected to have high momentum, likely more than 100~GeV/c. Even at this high momentum, HSCP would be traveling at a speed
appreciably slower than the speed of light due to their large mass. All of the long-lived SM particles produced at the LHC would be traveling at very nearly
the speed of light, $c$, even at a momentum of 10~GeV/$c$.
The slow speed of the HSCP means that it would arrive to the outer portion of CMS later than SM particles would.
To observe this, it must be possible to measure the arrival time of hits in the outer portions of CMS. This time measurement is important not only
for HSCP but also for SM particles to associate them with the correct LHC beam collision and to separate out SM particles not coming from
LHC collisions, such as from cosmic rays.

Chapter~\ref{sec:theory} of this work discusses the SM and a few models of physics beyond the SM. 
An emphasis is placed on theories which include new long-lived charged particles.
%the theoretical basis of particle physics is discussed. This includes the
%current theoretical framework that is used to describe the known particles and forces, the Standard Model (SM).
%Additionally, theories of possible physics beyond
%the SM are considered. An emphasis is placed on theories which include new long-lived charged particles as 
%a search for these particles form the basis of this work.
In Chapter~\ref{sec:app}, the experimental apparatus used in this work is presented. This includes the LHC and 
CMS, in particular parts of the apparatus especially important
for searches for long-lived charged particles are given extra detail.
Chapter~\ref{sec:timing} discusses the measurement of the arrival time of hits in the outer portion of CMS.
%The timing performance is shown in both the low latency online environment and the offline determination of the timing of a particle.
Chapter~\ref{sec:search} details four complementary searches for new heavy long-lived charged particles.
%using data collected with CMS from collisions at the LHC.
%The different searches focus on different signatures new particles could leave in CMS depending their properties and how they interact with CMS
%The searches were performed with
%the CMS collaboration and are in the process of being submitted to a journal, along with one more search. One of the searches
%is based almost entirely on my own work while I was one of the lead analyzers in a small group for the other three.
Concluding remarks on the presented work are given in Chapter~\ref{sec:conclusion}.

\chapter{Relevant Theoretical Issues \label{sec:theory}}

\section{Introduction}
Our understanding of particles and how they interact is constantly evolving through new experimental results and theoretical breakthroughs.
In this chapter, the current theoretical framework in particle physics is briefly described. Also, theories of physics beyond this framework are introduced,
with a focus on theories which predict the existence of new long-lived charged particles.

\section{Standard Model \label{sec:SM}}
The Standard Model (SM) of particle physics is a framework for describing fundamental and composite particles and the forces that govern how they interact. 
More information about the Standard Model can be found in~\cite{Srednicki_2007, griffiths2008introduction}

Particles in the SM are split into bosons, which have integer spin, and fermions, which have half integer spin.
The bosons are the carriers of the forces of the SM while the fermions act as the matter fields.

%The forces in the SM are the electromagnetic, strong, and weak forces. The carrier of the strong force is the 

Formally, the SM is described according to the symmetry group
\begin{equation}
SU(3)_C \times SU(2)_L \times U(1)_Y
\label{eq:SMGroups}
\end{equation}
where SU(N) denotes special unitary groups of dimension N. 
%Special unitary groups have the property of having a determinant of one.
Reading from left to right the groups represent color, weak isospin, and hypercharge. 
The color symmetry group is responsible for the strong force which is carried by the gluon.
A mixture of the weak isospin and hypercharge groups are responsible for the weak and electromagnetic forces. Left handed fermions are doublets of the
isospin group while the right handed fermions are singlets as they do not interact with the raising and lowering operators of the SU(2) group.
The weak force is carried by the W and Z bosons while the electromagnetic force is carried by the photon.

The last of the bosons in the SM is the Higgs Boson. The Higgs Boson is a scalar which allows it to have a non-zero vacuum expectation value (VEV).
This non-zero VEV breaks the electroweak symmetry and results in the W and Z bosons acquiring mass while the photon remains massless.
The Higgs Boson's non-zero VEV also gives mass to the fermions by allowing for a coupling between the singlet right handed fermions and doublet left handed fermions
which is otherwise not allowed.
%Describe the Higgs mechanism in this paragraph, don't really know how to do it currently.
In June 2012, the CMS and ATLAS experiments at the Large Hadron Collider
(see Ch.~\ref{sec:app}) announced the discovery of a new particle with properties similar to the Higgs Boson. If the particle is confirmed
to be the Higgs Boson, this would represent the final particle of the SM to be discovered.

The fermions in the standard model are split between quarks and leptons. Both the quarks and leptons are arranged into three families
with each family containing two quarks and two leptons. 
%Figure~\ref{fig:SM} (source CERN) shows all the confirmed particles in the SM along with their mass, electric charge, and spin.
The first family contains the up and down quarks, the electron, and the electron neutrino.
The second family has the strange and charm quarks, the muon, and the muon neutrino. The third family contains the bottom and top quarks,
the tau, and the tau neutrino. The down, strange, and bottom quarks have charge -1e/3 while the up, charm, and top quarks have charge +2e/3.
The electron, muon, and tau have charge -1e and all of the neutrinos are electrically neutral. All of the electrically charged particles in the SM have
an antiparticle with the opposite charge.

%\begin{figure}
%  \begin{center}
%      \includegraphics[clip=true, trim=0.0cm 0cm 0.0cm 0cm, width=1.0\textwidth]{figures/SM}\\
%      {Confirmed particles in the SM. Source: CERN
%        }
%      \label{fig:SM}
%  \end{center}
%\end{figure}

The quarks and gluons have color charge and interact through the strong force. There are three copies of each quark for each of the three different color charges.
The strength of the strong force increases with distance leading to color confinement where no free particles can exist with color charge.
Thus all quarks and gluons will form composite particles that are color neutral, called hadrons. There are three combinations of particles that lead to color
neutral composites. The first is a quark and an anti-quark referred to as mesons such as a charged pion which is made of an up quark and and down antiquark.
The second is three quarks (or anti-quarks) referred to as baryons and includes the proton and neutron.
Mesons and baryons can be either electrically charged or neutral.
The third is a pair of gluons, this type of particle is theoretically allowed but has never been experimentally observed. It would always be electrically neutral.

When quarks and gluons are produced at high energies, the binding of the strong force normally results in numerous secondary quarks and antiquarks being produced.
These quarks and antiquarks will then form hadrons in a process called hadronization.
%form composite particles after being created at particle colliders like the LHC, they will normally create a large number of particles, called hadronization. 
All of the hadrons will be traveling in roughly the same direction resulting in a stream of colinear particles from the production point.
%The result is a large number of nearly colinear particles emanating from the interaction point. 
This beam of partilces is referred to as a jet.

Most particles in the SM, both fundamental and composite, have very short lifetimes preventing them from being directly detected. The only stable particles
in the SM are the electron, proton, photon, and the neutrinos. Additionally, a free neutron has a lifetime of eight minutes but it may be stable inside a nucleus.
A few other particles have lifetimes long enough to be detected before decaying including the muon, pion, and kaon.

The proton-proton collisions at the LHC will create vast numbers of these SM particles. As the strong force has the largest coupling,
a large majority of the events will be the production of light quarks and gluons. A small fraction of the events, though a large total number given the very
high rate of collisions at the LHC, will produce particles such as W and Z bosons, top quarks, and possibly, if it is confirmed, the Higgs Boson.
These particles will quickly decay into other SM particles like muons, electrons, and b quarks. This production of stable and long lived SM particles,
particularly muons, form part of the background for the search for HSCP detailed in Chapter~\ref{sec:search}.

%\subsection{Cosmic rays \label{sec:cosmics}}
The production of SM particles proceeds not only in the proton-proton collisions at the LHC but also through astrophysical processes.
The earth is constantly being bombarded with high momentum protons from astronomical sources. These protons interact with the earth's atmosphere
resulting in the production of numerous charged and neutral pions. Charged pions will then decay into high momentum muons.
As the muons will have a large relativistic boost, they will be able to reach the earth before decaying.
%At sea level the flux of muons is approximately one per 10$cm^2$.
Muons lose only a small amount of energy in interactions with matter,
allowing the highest energy cosmic ray muons to penetrate through large amounts of earth.
This allows cosmic ray muons to reach the detectors of the LHC potentially creating a background for searches for new physics.

\section{Beyond Standard Model Theories and Heavy Stable Charged Particles \label{sec:BSM}}
More information about theories beyond the SM and heavy stable charged particles can be found in~\cite{Fairbairn:2006gg, Martin:1997ns, Tata:1997uf}.

While the SM has proven to be a very robust theory, there are numerous reasons to believe it is not complete. The reasons for this include runaway
radiative corrections and large amounts of fine tuning of parameters. At a minimum, the theory must be replaced at the Planck energy scale ($10^{18} GeV$)
where the gravitational force becomes as strong as the other forces. To address issues like these numerous
theories have been put forth for physics beyond the SM (BSM). 
If these BSM theories are accurate, evidence of them would likely be present in the high energy collisions produced at the LHC.
Some of these BSM theories predict the existence of heavy meta-stable
charged particles (HSCP) with lifetimes greater than a few ns, long enough to traverse the length of typical particle detectors. 

One of the most popular BSM theories is supersymmetry (SUSY). In SUSY, a new symmetry is added to the SM which gives
each SM particle a superpartner particle with spin different by one half. 
The names of the SUSY particles are generally found by prepending an s (for scalar) to the SUSY partners of spin 1/2 particles,
so the SUSY partner of the electron is the selectron, and adding ino to the end of other particles, so the higgs SUSY partner is the higgsino.
As no SUSY particles have yet been discovered the
symmetry must be broken at some scale giving the SUSY particles masses larger than SM particles. In order to address the unresolved issues in the SM, this mass gap is
expected to be no larger than about 1 TeV. In addition to adding a new symmetry, SUSY also predicts the existence of a new multiplicatively conserved quantity called R-parity.
SUSY particles have an R-parity value of -1 while SM particles have a value of 1. This implies that the lightest SUSY particle (LSP) will be stable and in most
SUSY theories it is taken to be electrically and color neutral so as to be the astronomically observed dark matter.

Other SUSY particles besides the LSP could have a long lifetime in certain areas of SUSY parameter space. In the minimal supersymmetric standard model (MSSM) the LSP is the
neutralino (superpartrner of a neutrino) in most cases. The next lightest SUSY particle (NLSP) can be long-lived if the mass splitting between the NLSP and the LSP is small, this
can happen for many different particles as the NLSP. 
Non-universal squark masses can be used to make the mass difference between the stop and the neutralino too small for the stop decay to a neutralino and
a bottom quark to be kinematically allowed.
%One case of interest is that of the scalar top (stop $\tilde{t}$) as the NLSP motivated by electroweak
%baryogenesis~\cite{Balazs:2004bu}. If the mass difference between the stop and the neutralino makes the stop decay to a neutralino and a b quark kinematically not
%allowed as arranged by non-universal squark masses, 
Then the stop decay happens via the radiative decay to a charm quark and neutralino making the stop very long-lived.

Another variant of supersymmetry is split SUSY. In split SUSY, scalar SUSY particles have very large masses while other particles remain at the TeV scale.
As gluinos $\tilde{g}$ (superpartners of the gluon) must decay through squarks this can make the gluino quite long-lived.

Gluinos and stops have color charge and as such will form composite hadrons with SM quarks and gluons after production, referred to as $R-hadrons$.
These $R-hadrons$ can be mesons, baryons, or, for gluinos, a glue-ball made of a gluino and a SM gluon.
%Examples of R-hadrons are shown in Fig.~\ref{fig:Composites}.
$R-hadrons$ can be electrically neutral or have charge Q, taken here and everywhere else in this paper unless otherwise stated as the absolute value of the charge,
of 1e or 2e, where e is the charge of the electron.
One particularly interesting case is for glue-balls which will always be electrically neutral. The fraction of gluinos forming glue-balls is
a free parameter in the theory. If the fraction is 100\% then all gluino $R-hadrons$ will be produced electrically neutral.

%\begin{figure}
%  \begin{center}
%      \includegraphics[clip=true, trim=0.0cm 0cm 0.0cm 0cm, width=1.0\textwidth]{figures/quarks}\\
%      {Different types of R-hadrons that color charged HSCP can form. Top left: Stop HSCP baryon with two SM quarks.
%Top right: Stop HSCP meson with a SM anti-quark. Bottom left: Gluino HSCP baryon with three SM quarks. Bottom right: Gluino HSCP glue-ball with a SM gluon.
%	}
%      \label{fig:Composites}
%  \end{center}
%\end{figure}

After the $R-hadrons$ are produced at the LHC, they will propagate out to and interact with the LHC detectors. In the interactions with the
detectors it is possible for the electrical charge of the $R-hadron$ to change, possibly going from neutral
to charged or from charged to neutral. The process occurs through an exchange of quarks with the detector material in nuclear interactions.
%an example of this can be seen in Fig~\ref{fig:Rhadron} taken from ref.~\cite{SMP}.
The modelling of these interactions has some uncertainty, of particular interest is the fraction of $R-hadrons$ that will be electrically charged after an interaction.
Two models are considered in this work.
The first is the model presented in~\cite{Kraan:2004tz, Mackeprang:2006gx} 
which is referred to as the cloud model where the $R-hadron$ is pictured as a spectator HSCP surrounded by a cloud of light, color-charged SM quarks and gluons.
This model results in a mixture of neutral and charged $R-hadrons$ after a nuclear interaction.
The second model, referred to as charge-suppressed, results in all $R-hadrons$ becoming neutral after a nuclear interaction as described in~\cite{Mackeprang:2009ad}.
%Most HSCP will not have a nuclear interaction while passing through the CMS tracker however a very large majority will have one in the calorimeter system.

%\begin{figure}
%  \begin{center}
%      \includegraphics[clip=true, trim=0.0cm 0cm 0.0cm 0cm, width=1.0\textwidth]{figures/Rhadron}\\
%      {R-hadron interaction
%	}
%      \label{fig:Rhadron}
%  \end{center}
%\end{figure}

A third possible HSCP in SUSY  is the production of long-lived staus $\tilde{\tau}$ in gauge mediated symmetry breaking (GMSB) SUSY~\cite{Giudice:1998bp}. 
In GMSB, the gravitino is very light and almost always the LSP.
GMSB models are characterized by six parameters which determine the mass heirarchy and decays of SUSY particles.
One of these parameters is the number N of SU(5) chiral multiplets added to the model which act as ``messengers''.
As long as N is not too small the NLSP is likely to be the stau.

The lifetime of the stau is given by~\cite{Fairbairn:2006gg}
\begin{equation}
\tau_{Stau} = 0.1 \left(\frac{100 GeV}{m_{Stau}}\right)^5 \left(\frac{m_{\tilde{G}}}{2.4 eV}\right) mm/c
\label{eq:lifetime}
\end{equation}
with $m_{\tilde{G}}$, the gravitino mass, set by
\begin{equation}
m_{\tilde{G}} = 2.4 c_{Grav} \left(\frac{\sqrt{M\Lambda}}{100 TeV}\right)^2 eV
\label{eq:gravmass}
\end{equation}
with $c_{Grav}$ being the ratio between the fundamental SUSY breaking scale and the effective one felt by the messenger particles,
$\Lambda$ the effective SUSY breaking scale, and M the mass of the messengers.
The parameter $C_{Grav}$ relates to how the SUSY breaking is transmitted to the messengers, if the communication is done perturbatively then $c_{Grav}$
will be very large giving the stau a long lifetime.

Other BSM theories besides SUSY can also contain HSCPs.
An interesting scenario for HSCP is the production of particles with charge not equal to $1e$.
One model that includes non-unit charged HSCP is the production of particles that are neutral under $SU(3)_C$ and $SU(2)_L$ 
but have electric charge meaning they only couple to the photon and Z boson through $U(1)_Y$ interactions~\cite{Langacker:2011db}.
%The electric charge Q of the particle is not constrained to be 1e.
%Note here and for the rest of this paper Q is taken as the absolute value of the electrical charge unless specifically stated otherwise.
The HSCP could be produced with fractional charge ($<1e$) or multiple charge ($>1e$).

%Another BSM theory that includes HSCP is Universal Extra Dimensions (UED). In UED SM fields, quarks, and gluons propagate through new dimensions with the
%known SM fields representing the ground state mode in the new dimensions. States in excited modes in the new dimension would be seen as new particles.
%A new quantum number is introduced in UED for momentum conservation in the extra dimensions that forces the lightest excited particle to be stable,
%making it a dark matter. The excited states of light SM particles could have long enough lifetimes to be experimentally detectable.



\chapter{Experimental Apparatus \label{sec:app}}

\section{Introduction}

The apparatus used for this paper is a combination of a large particle accelerator complex and a detector used to measure the results of particle collisions.
%r than how the term is normally used.  
Protons are accelerated and brought to a collision by the Large Hadron Collider (LHC)
located outside Geneva, Switzerland spanning the Swiss-French border. The protons are accelerated in smaller linear and cyclical accelerators before being injected
into the LHC. The protons are brought to a collision at four spots along the LHC. Surrounding one of these spots is the Compact Muon Solenoid (CMS) detector. The CMS
detector consists of multiple subsystems which work together to identify signatures of different types of particles.

\section{Large Hadron Collider \label{sec:LHC}}
A full description of the LHC can be found in~\cite{1748-0221-3-08-S08001}, a short summary is included here.
The LHC is a two-ring superconducting synchotron designed
to collide particles at high energy and high luminosity. It sits in a 26.7 km tunnel located 45-170m underneath the Swiss-French countryside outside of Geneva, Switzerland.
The LHC can create collisions with either protons or heavier ions. This leads to three possible operational modes, proton-proton, ion-ion, and proton-ion.
Only in proton-proton operational mode is there a possibility to discover HSCPs and it is the only mode discussed in this paper.

The LHC was designed to accelerate protons to an energy of seven TeV and collide them at a center of mass energy($\sqrt{s}$) of 
fourteen TeV with an instantaneous luminosity of $10^{34}cm^{-2}s^{-1}$. The protons are brought
to a collision at four points along the LHC beamline. Surrounding two of these interaction points sit the general purpose detectors of CMS and ATLAS. These detectors are 
meant to recieve the highest instantaneous luminosity the LHC can supply. The other interaction points are surrounded by the special purpose detectors LHCb and ALICE and
are designed to have instantaneous luminosities of $2\times10^{29}cm^{-2}s^{-1}$ and $10^{27}cm^{-2}s^{-1}$, respectively. This paper considers data collected by the
CMS detector.

The acceleration of protons to their final energy of 7 TeV is done in series of steps employing smaller accelerators located on the CERN campus. 
The protons originate in the linear accelerator Linac2 where they are passed through a series of synchotron accelerators, the Proton Synchotron Booster, the Proton Synchotron,
and the Super Proton Synchotron, with their energy raised to 1.4 GeV, 25 GeV, and 450 GeV, respectively. After passing through the Super Proton Synchotron the protons
are passed into the LHC. The LHC then accelerates the protons to their final design energy of 7 TeV.

The beams are designed to contain proton bunches spaced such that collisions at the interaction points occur every 25ns. 
%Running at the LHC thus far
%This time span sets the window of 
The LHC can hold a total of 2,808 bunches, in some places it is designed to have are gaps larger than 25ns between bunches to allow for dumping of the beam without harming the LHC.
Each 25ns time window is referred to as a bunch crossing window, whether there are proton bunches colliding in CMS or not.
Each collision between the proton bunches can
result in more than one proton-proton collision. This results in the the detectors around the LHC interaction points seeing numerous proton-proton collisions
overlayed on one another.
%The effect of this on the search for HSCPs is discussed in Section~\ref{sec:SystUnc}.

The commisioning of the LHC saw it run at a progression of lower energies building towards the design energy. In 2008 the LHC was run at
$\sqrt{s}=900$ GeV and for a short period at 2.36 TeV. Then after further work on the LHC, the energy was raised to 7 TeV for both 2010 and 2011 and then to 8 TeV in 2012.
This paper only covers the data collected at 7 and 8 TeV in 2011 and 2012. It is planned to raise the energy to its design goal of 14 TeV through additional work on the
LHC and the injector system.

Similarly, the instantaneous luminosity was ramped up during the commisioning phase. During the 2012 running, the machine ran with the proton bunches separated by 50ns.
The instantaneous luminosity often reached $7\times10^{33}cm^{-2}s^{-1}$. With 50ns spacing this means that the per bunch luminosity actually exceeded
the design value. It is planned to run with 25ns bunch spacing in future LHC running.

\section{Compact Muon Solenoid}

The CMS detector is built around one of the interaction points of the LHC. A full description of CMS be found in 
references~\cite{Chatrchyan:2008zzk, Bayatian:922757}.

CMS was designed to be a general purpose detector that would have sensitivity to a wide range of physics. This is important for a search for HSCP as the detector
is used in ways not typically done in most CMS analyses.
The central feature of CMS is a superconducting solenoid magnet with a 6m diameter and 13m length that provides a 3.8T magnetic field. The return field from the solenoid
is powerful enough to saturate 1.5m of iron, this allows for a strong magnetic field to be present outside of the solenoid.
CMS has a cylindrical shape with an onion like design where inner subdetectors are nested inside of outer ones. From inside out these subdetectors are an all silicon
tracker, an electromagnetic and hadronic calorimeter, the magnet, and finally the muon system. The various subdetectors and their role in identifying
SM particles can be seen in Figs~\ref{fig:CMSPart} and~\ref{fig:CMSSlice}.

CMS employs a right handed coordinate system with the x-axis pointing to the center of the LHC ring, the y-axis pointing vertically upward, and thus making the z-axis
be along the beam line pointing in the clockwise direction if looking at the LHC from above. The azimuthal angle, $\theta$, is defined relative to the z-axis. The variable
psuedorapidity, $\eta$, is defined as $\eta = -\ln{[\tan{(\theta/2)}]}$. The polar angle, $\phi$, is defined relative to the x-axis, meaning that vertically
upward (downward) has a $\phi$ value of $\pi/2$ ($-\pi/2$).

\begin{figure}
  \begin{center}
      \includegraphics[clip=true, trim=0.0cm 0cm 3.0cm 0cm, width=0.9\textwidth]{figures/apparatus/CMS_LongView_noME42.pdf}
        \caption[Cross-section view of CMS detector]
        {Cross section of CMS detector. Inner silicon track bottom right in green, electromagnetic and hadronic calorimeter in light gray and yellow, respectively.
Muon detectors on the outside in blue.
         }
      \label{fig:CMSPart}
  \end{center}
\end{figure}

\begin{figure}
  \begin{center}
      \includegraphics[clip=true, trim=0.0cm 0cm 3.0cm 0cm, width=\textwidth]{figures/apparatus/CMS_Slice.png}
        \caption{Expected interactions of SM particles as they propagate through CMS.
        }
      \label{fig:CMSSlice}
  \end{center}
\end{figure}

The possibility that a particle containing an HSCP can interact with the detector and change its charge
means that it may not look like any of the particles in Figure~\ref{fig:CMSSlice}. The particle
may be produced neutral and only gain charge as it passes through the calorimeter. The only record of its hits will be in the muon system giving the signature shown in
Figure~\ref{fig:CMSMuOnly}. In addition, the particle may be produced charged charged but then become neutral after interacting with CMS giving the signature shown in
Figure~\ref{fig:CMSTkOnly}. To discover HSCP with these exotic signatures it is necessary to conduct dedicated searches.
Searches of this type are presented in Chapter~\ref{sec:search}.



\begin{figure}
  \begin{center}
      \includegraphics[clip=true, trim=0.0cm 0cm 3.0cm 0cm, width=0.9\textwidth]{figures/apparatus/ParticleInCMS_0009_Becoming_Charged}
      \caption[HSCP produced neutral and only becoming charged after interacting with the CMS detector.]
        {HSCP produced neutral and only becoming charged after interacting with the CMS detector.
	 }
      \label{fig:CMSMuOnly}
  \end{center}
\end{figure}

\begin{figure}
  \begin{center}
      \includegraphics[clip=true, trim=0.0cm 0cm 3.0cm 0cm, width=0.9\textwidth]{figures/apparatus/ParticleInCMS_0000_HSCP-(becoming-neutral).png}
      \caption[HSCP produced charged and becoming neutral after interacting with the CMS detector.]
        {HSCP produced charged and only becoming neutral after interacting with the CMS detector.
         }
      \label{fig:CMSTkOnly}
  \end{center}
\end{figure}

\subsection{Subdetectors \label{sec:subsystems}}
The innermost part of CMS is an all silicon tracker. Closest to the interaction point are pixel detectors with three barrel layers and two endcap disks, totalling 1,440 modules. 
Outside of this are strip detectors with ten barrel layers and twelve endcap disks. It extends up to a psuedorapidity range of 2.5 with the resolution on track
$p_T$ being approxiamtely 1.5\% for a 100 GeV$/c$ particle at $|\eta| = 1.6$ and growing larger at high $|\eta|$ due to the decreased lever arm. Both the strips and the
pixels have an analog readout of the deposited charge with a maximum readout of roughly three times the charge expected to be deposited by a muon. Charge from
particles traversing the inner tracker is expected to be spread out among multiple modules in the same layer allowing the position of the particle to be calculated
more precisely then simply the center of the module. The charge sharing also allows the possibility to identify hits where two particles have overlapped.

Outside of the inner tracker is the calorimeter. The purpose of the calorimeter is to measure the energy of particles and aid in their identification by stopping
particles at different points in the calorimeter.  The calorimeter is split into an inner electromagnetic calorimeter (ECAL) and an outer hadronic calorimeter (HCAL). 
The ECAL is made of 75,848 lead tungstate ($PBWO_4$) crystals split between the barrel and endcap. As particles lose energy in the ECAL
the crystals emit scintillation light which is collected by photodetectors. The HCAL consists of plates of brass absorbers interleaved with scintillator
detectors.  Electrons and photons are likely to stop in the ECAL where they deposit all of their energies. Hadrons, electrically charged or neutral, will
deposit some energy in the ECAL but will deposit most in the HCAL where they are very likely to come to a rest. Muons will deposit of the order of two GeV of energy in
the calorimeter and are generally the only charged SM particles that are able to exit the calorimeter.

The outermost part of the detector is the muon system which is split into three parts, Cathode Strip Chambers(CSC), Drift Tubes(DT), and Resistive Plate Chambers(RPC).
The CSC cover the forward part of the detector with $|\eta|>0.9$ while the DT and the RPC cover the barrel portion extending up to $|\eta|$ of 1.2 and 1.6, respectively.
The muon system is comprised of four stations of chambers with the lead for the magnet return yoke located between the stations. The magnet return yoke provides a magnetic field
in the muon system.

CSC chambers have a trapezoidal shape with six layers of cathode strips and anode wires arranged in a nearly orthogonal pattern. 
The strips run radially away from the beam line and measure the $\phi$ of hits while the wires measure the radial position of hits. Charge collected on the wires
is passed to a constant fraction discriminator which outputs a 40ns pulse. The pulse is sampled every 25ns and this sampling is readout. The amount of charge on the strips
is readout every 50ns. The charge is used offline to get a more precise estimate of the position and time of the hit.
The CSCs are layed out with four stations with increasing $z$ from the interaction point and rings of increasing radial distance from the beam line.

DT chambers have two or three superlayers which themselves are composed of four layers of drift cells which are staggered by half a cell. All of the DT chambers have
two superlayers oriented parallel to the beam line, these superlayers measure the position of particles in the $r-\phi$ plane.
The three inner stations additionally have a superlayer running perpendicular to the beamline to measure the position of particles in the $r-z$ plane.

RPC chambers are gaseous parallel plate detectors that can provide a time resolution of 2ns, which is much smaller than the design LHC bunch spacing of 25ns allowing for
a very high efficiency to correctly tag hits with the correct event. The spatial resolution is sufficient to be able to associate RPC hits with hits
from the other muon subdetectors.

\subsection{Computing \label{sec:computing}}
The rate of proton-proton collisions inside of CMS is too large for all of them to be readout and stored offline. To deal with this CMS employs a two level trigger
that selects interesting events online. The level one (L1) trigger must reduce the rate of events readout to less than 100 kHz in less than 3$\mu s$
requiring a completely firmware based approach. Events are selected by a variety of algorithms but most of them look for a high momentum track in the muon
system, large amount of energy in the ECAL or HCAL, or a combination of these. Signals from these systems trigger the readout of the rest of detector through
the data acquisition system. As the LHC was designed to operate with 25 ns bunch spacing many of the subsystems, the tracker especially, only readout
the data in the 25ns window associated with the event. This means that triggers that pre or post-fire will not contain much of the data from the event.
This can be issue for HSCP that are travelling so slowly that they reach the muon system in the time window associated with the next bunch crossing window.
However, a special configuration of the RPC trigger exploits the fact that current running of the LHC has been done with at least 50ns spacing

All hits in the RPC are sent to the trigger electronics twice, once for the bunch crossing window they are associated with and also for the one proceeding it.
From there the trigger electronics treat the advanced RPC hits in the same manner as they do all other hits. 
This allows the RPCs to trigger the readout of the event preceeding the arrival of the particle in the RPCs.
This means that HSCP which arrive to the muon system up to 37.5ns after
a muon is expected could still trigger the readout of the correct data in the rest of the detector.
% corresponding with the bunch crossing window it was created in.

%Despite the RPC signaling to readout two events only one will ever be actually collected as readout of consecutive events is forbidden by the DAQ. 
To ensure collision muons still maintain the correct behavior, accept signals sent for the bunch crossing window immediately
preceeding a bunch crossing window with protons passing through CMS are rejected. 
So signals from collision muons will attempt to pretrigger but this will be vetoed and the following event will be correctly readout.
%A schema illustrating this behavior is shown in~\ref{fig:RPC_HSCP}. 
This configuration is only possible  when proton collsions are spaced by at least 50ns so that accept signals from successive 25ns
bunch crossing windows can be unambiguously classified.

%\begin{figure}
% \begin{center}
%  \includegraphics[clip=true, trim=0.0cm 0cm 0.0cm 0cm, width=1.0\linewidth]{figures/RPC_HSCP}
% \end{center}
% \caption{Way too long... Schema of a~temporal behavior of the RPC trigger and
%its influence on the data read by the data acquisition system.
%Time measured in BX quanta goes from left to right.
%The results of appearance
%of three (separate in time) objects of different type is shown:
%pp collision
%muon, HSCP which is delayed by one BX at the exit from the muon system and
%outward going cosmic muon delayed by one BX with respect to muons
%from pp collision.
%Each hit (read bolts) in the RPC is advanced by one BX and duplicated
%(blue squares) in the PAC
%(Pattern Comparator, chip in which main part of the RPC trigger
%logic is implemented). Only first and last RPC layers are shown.
%A~coincidence (pattern) of bits in the PAC in the same BX gives L1 trigger
%(blue arrows). But to obtain the HLT trigger a coincidence of
%the RPC L1 trigger with BPTX bit is required (yellow arrows).
%DAQ reads tracker data from HLT selected BX and RPC data from adjacent BXs
%(green rectangles). Both pp mouns and (not too slow)
%HSCPs give HLT trigger.
%Outward cosmic muons, which are late by one BX, also give HLT trigger
%(rightmost yellow arrow), but its tracker hits are not selected.
%Such cosmic muons will not become global muons.}
%   \label{fig:RPC_HSCP}
%\end{figure}

The next step in the trigger is the High Level Trigger (HLT) which must reduce the number of events to a few hundred Hz on the order of a second. The HLT
is software based and there are a wide variety of algorithms used to identify interesting events and store them for offline analysis. The HLT is split into two different
phases, Level 2 (L2), and Level 3 (L3). The L2 step is mostly concerned with confirming the L1 decision and reducing the rate so that higher level objects
can be built within the time restrictions. The L3 step builds these objects, often reconstructing tracks of particles in the inner tracker and matching them
to objects in other parts of the detector, and then applies requirements on the objects selecting which events to pass for storage at computers located at CERN
and throughout the world.

CMS maintains a software, CMSSW, which is responsible for taking the raw data readout from CMS and reconstructing what was happening in the event.
This includes applying calibration constants, finding tracks, and indentifying particles.
%attempts to reconstruct the particles in the event, identify them as one of the long-lived SM particles, give 
%a multitude of information about the particle, and apply any necessary calibration constants. 
%The code also calculates event level quantities such as the total momentum of all the particles in the event. 
After this reconstruction, the data size is at the scale of petabytes which is too large for offline analyzers to run over frequently. 
To deal with this copies of the data are produced dropping lower level quantities and selecting only events that a particular analysis is interested in studying.

CMSSW is also tasked with simulating how particles, coming from both SM processes and new physics, would interact with the detector so that this can be used to
compare against data. Two steps are performed before the simulation has the same format as data readout from the detector, at that
point it follows the same chain as data. The first step is the simulation of the proton-proton collision
and the particles that are created from it, the detector is not used at all in this step. The next step is the simulation of the interaction of these particles with the detector
and the behavior of the detector electronics, including the L1 trigger. After this point, the simulation is handled the same as data.

\chapter{Muon System Timing \label{sec:timing}}

\section{Foreword}
This chapter details the measurement of the arrival time of particles in the muon system of CMS. A particular focus is put on the measurement in the CSC system
with a description of the measurements in the whole of the muon system given at the end. 

\section{Introduction}
Muons coming from collisions in the LHC take approxiamtely 25-40ns to travel from the interaction point to the muon system. As CMS was designed to collect data with protons
colliding every 25ns the time of flight (TOF) of muons is a significant time interval. The muon system must be able to associate tracks in the system to the
correct bunch crossing window for the L1 trigger.
This is required to trigger the readout of the data in the rest of the detector associated with the collsion that the track came from.
The method to determine timing synchronization of the CSC subsystem is described below.

Additionally, the timing in the muon system can be used to separate out different sources of tracks. These sources include collision muons from the
triggered bunch crossing wondow, muons from adjacent bunch crossing windows, muons from cosmic rays, and possibly 
HSCPs predicted in theories of new physics. To do this the time of hits in the muon system is measured and a combined time for each track is calculated.

\section{CSC Hit Timing}
Hits in the CSCs are found from a combination of signals from the anode wires and cathode strips. Both of the signals can be used to estimate the time of the hits.

Time is measured by the cathode strips in two ways, one for online use in the L1 trigger and one for offline measurement. The online measurement finds the peak of the
charge distribution and associates it with the particular bunch crossing window. For offline determination of the position and 
time of the hits, the charge on cathode strips is sampled every 50ns.
The time of the hits is estimated with a fit to the charge distribution. Calibration constants are subtracted from the times during reconstruction to center the times at zero.
The constants are found for each chamber and are derived from times associated with high quality, high momentum muons. Cathode times have a resolution of approxiamtely 7.0ns.

As stated in Section~\ref{sec:subsystems}, signals from the anode wires are passed to a constant fraction discriminator which outputs a 40ns pulse
that is then digitized every 25ns. Depending on when the pulse starts, the hit can have either one or two bits being high. Given the same
first high bit, it can be inferred that hits with the next bit low arrived earlier than hits with the next bit high.
Hits with only one high bit are estimated to have arrived at the time of that bit while those with two high are estiamted to be from the average of the two bits.
Thus, it is possible to estimate the time of anode hits with a 12.5 ns quantization. The anode times are calibrated to have a mean of zero in the same method as
per the cathode times. The resolution of the anode hit timing is approximately 8.6 ns 

The distribution of the time of anode and cathode hits associated with high quality, high momentum muons is shown in Fig~\ref{fig:hittime}.
As can be seen in the right plot the anode time has a large tail of positive times. This is dealt with by a cleaning procedure defined below.

\begin{figure}
  \begin{center}
      \includegraphics[clip=true, trim=0.0cm 0cm 2.0cm 0cm, width=0.44\textwidth]{figures/timing/CathodeTime}
      \includegraphics[clip=true, trim=0.0cm 0cm 2.0cm 0cm, width=0.44\textwidth]{figures/timing/AnodeTime} \\
      \includegraphics[clip=true, trim=0.0cm 0cm 2.0cm 0cm, width=0.44\textwidth]{figures/timing/CathodeTimeLog}
      \includegraphics[clip=true, trim=0.0cm 0cm 2.0cm 0cm, width=0.44\textwidth]{figures/timing/AnodeTimeLog} \\
      \caption[Distribution of time of anode and cathode hits]
      {Left column: Distribution of cathode time of hits. Right column: Distribution of anode time of hits. Top row: Linear scale. Bottom row: Log scale.
	}
      \label{fig:hittime}
  \end{center}
\end{figure}

The anode and cathode hits in a chamber are used to reconstruct a segment which is meant to represent the passage of the particle through the chamber. A time is
associated with the segments by averaging the anode and cathode times associated with the segments. The times are weighted by one over their variance.
To remove the large tail in the anode time measurement,
a cleaning procedure is applied to the anode times to remove outlier hits. The procedure removes anode times more than three
sigma different from the average. The times of segments associated with high quality muons is shown in Fig.~\ref{fig:SegTimes}. The resolution on the segment times is 3.0ns.

\begin{figure}
  \begin{center}
      \includegraphics[width=0.44\textwidth]{figures/timing/StripAndWireSegmentTime}
      \caption[Distribution of times of segments associated with high quality muons]
      {Times of segments associated with high quality muons.
        }
      \label{fig:SegTimes}
  \end{center}
\end{figure}

\section{CSC Trigger Timing}

The CSCs are a key component of the L1 trigger system and it is important that they associate tracks in the system with the correct LHC bunch crossing window.
The CSCs build tracks for the L1 trigger with the CSC Track Finder (CSCTF) by combining track stubs coming from the CSC chambers. 
The stubs are associated with a particular bunch crossing window and
the track finder uses majority logic of the stubs used to build the track to associate the track with a bunch crossing window. In cases where there are an equal number
of stubs from different bunch crossing windows, say two track stubs coming from adjacent bunch crossing windows, the CSCTF preferentially selects the later bunch crossing window.

As mentioned in Section~\ref{sec:subsystems}, there are six layers of cathode strips and anode wires in a CSC chamber.
Electronics on the chamber collect hits from the cathode strips and anode wires and separately create trigger primitives called Cathode Local Charged Track (CLCT)
and Anode Local Charged Track (ALCT), respectively. The two separate trigger primitives are then combined to form a Local Charged Track (LCT). The trigger primitives
must be associated with events within three bunch crossing windows of one another to be combined.
The bunch crossing window that the LCT is associated with is set by the ALCT.

The timing of the ALCT is determined by the timing of the third anode hit, its first high bit, 
to arrive to the ALCT circuit board. A common offset per chamber can be applied to the
anode hits to give the best timing synchronization of the ALCTs. To determine the offset the arrival time of the anode hits is studied offline using hits from
high quality muons. The average time of the anode
hits can be correlated with the probability for a chamber to produce an ALCT in the bunch crossing window before it should (pretrigger) and after it should (posttrigger).
The offsets for each chamber can be tuned to give an expected pretrigger and posttrigger probability.

However, the CSCTF logic means that simply setting the offset to give an equal probability to pretrigger and posttrigger is not optimal. This can be seen by looking at
the case where the CSCTF only receives two track stubs, this is also the case where the CSC online timing is most important. If the CSCTF receives one LCT in the bunch
crossing window before the collision and one in the correct bunch crossing window, it will preferentially choose the later LCT and associate the combined track with the correct
bunch crossing window.  In order to pretrigger the readout of the event more than one LCT must arrive early.
On the other hand, if it recieves one LCT in the correct bunch crossing window and one in the proceeding bunch crossing window, the track will be associated with the
bunch crossing window following the collision. Thus, the probability to pretrigger the event can be written as $P_{LCTPre}^2$ while the posttrigger probability can be written as
$2 \times P_{LCTPost} - P_{LCTPost}^2$. 

Figure~\ref{fig:AnodevsprePost} shows the probabilities to pretrigger and posttrigger both at the chamber level and the expected probability at the CSCTF versus
the average anode time of a chamber. The chambers are split into three categories depending on which station and ring they belong to. One category is chambers in the
first ring and station, another the chambers in the first ring not in the first station, and the last those not in the first ring. The design of these chambers are all
slightly different so it is allowed for them to have different optimal times.

\begin{figure}
  \begin{center}
      \includegraphics[clip=true, trim=0.0cm 0cm 3.0cm 0cm, width=0.44\textwidth]{figures/timing/ME11_Anode_vs_all3}
      \includegraphics[clip=true, trim=0.0cm 0cm 3.0cm 0cm, width=0.44\textwidth]{figures/timing/ME11_Anode_vs_TF_all} \\
      \includegraphics[clip=true, trim=0.0cm 0cm 3.0cm 0cm, width=0.44\textwidth]{figures/timing/Ring1_not11_Anode_vs_all3}
      \includegraphics[clip=true, trim=0.0cm 0cm 3.0cm 0cm, width=0.44\textwidth]{figures/timing/Ring1_not11_Anode_vs_TF_all} \\
      \includegraphics[clip=true, trim=0.0cm 0cm 3.0cm 0cm, width=0.44\textwidth]{figures/timing/Ring2_Anode_vs_all3}
      \includegraphics[clip=true, trim=0.0cm 0cm 3.0cm 0cm, width=0.44\textwidth]{figures/timing/Ring2_Anode_vs_TF_all} \\
      \caption[CSC pretriggering and posttriggering versus average anode time for LCTs and expected behavior at CSCTF]
      {Pretriggering and posttriggering versus average anode time. Left column shows the pretriggering and posttriggering for LCTs.
Right column shows what would be expected at the CSCTF. Top row for chambers in the innermost ring and station. Middle row is for all other chambers in the innermost ring.
Last row is for all other chambers.
        }
      \label{fig:AnodevsprePost}
  \end{center}
\end{figure}

From these plots an optimal value of 204ns is chosen for the chambers in the first ring not in the first station and 205ns for all other chambers.
The plot of the pretriggering and posttriggering in the track finder somewhat implies an earlier optimal but these are not used for two reasons.
The first is that pretriggering grows as the square of the pretriggering probability and since, as described below, the offsets can not be set exactly, chambers
that are slightly below optimal could lead to significant pretriggering in those chambers. Second, pretriggering prevents the readout of the collision event even by
another portion of the detector, CMS can not readout two consecutive events, while posttriggering does not have this issue. For these reasons slightly later times
that still have very low posttriggering are used.

The offsets can be moved in roughly 2 ns steps in the chamber firmware with the actual number possibly being different chamber to chamber. Shifting the offsets is a
somewhat complicated procedure and carries the risk of accidentally shifting the timing of a chamber by a large amount. Thus, the offsets are changed only
when deemed necessary, numerous iterations to get a perfect synchronization are not done. The synchronization with respect to the optimal values for all chambers
is shown in Fig.~\ref{fig:average_anodes}, most of the chambers are within one ns of the optimal time with none more than three ns off.

\begin{figure}
  \begin{center}
      \includegraphics[clip=true, trim=0.0cm 0cm 3.0cm 0cm, width=0.44\textwidth]{figures/timing/average_anodes}
      \caption[Average anode time of chambers relative to optimal values.]
      {Average anode time of chambers relative to optimal values.
        }
      \label{fig:average_anodes}
  \end{center}
\end{figure}

After this synchronization procedure is performed the timing of the LCTs is very good. This can be seen in Fig.~\ref{fig:ALCTBX} which shows the bunch crossing window
assigned to LCT matched to high quality muons. The distribution is purposefully made asymmetric to account for the CSCTF logic used further downstream.
The efficiency is 99\%, better than the 92\% design requirement.

\begin{figure}
  \begin{center}
      \includegraphics[clip=true, width=0.44\textwidth]{figures/timing/ALCT_Bx}
      \caption[Fraction of LCTs versus LCT bunch crossing window assignment relative to collision event]
      {Fraction of LCTs versus LCT bunch cross assignment relative to collision event
        }
      \label{fig:ALCTBX}
  \end{center}
\end{figure}

\section{Muon Track Timing}

Tracks, meant to represent muons or other particles passing through the detector, are built in the muon system connecting together the hits in the different chambers of
the CSCs, DTs, and RPCs. Numerous different timing quantities about the track are calculated under three different assumptions on how the particle
travels between the interaction point and the muon system. Only time measurements from the CSCs and DTs are used to calculate the timing quantities.

A particle of speed, v, travelling from the interaction point, will arrive at a location d in the muon system at
%~\ref{eq:speed}

\begin{equation}
t = d/v + t_0
\label{eq:speed}
\end{equation}

where $t_0$ is an overall offset. When the local timing variables were defined, they were calibrated such that a speed of light, c, particle would have an average time
of zero. Thus $d/c$ has already been subtracted from the times so the same quantity must be subtracted from the right hand side of~\ref{eq:speed}.
Additionally it is easier to work with \invbeta\ ($\equiv{c/v}$) instead of v. With these two ideas taken into mind Eq.~\ref{eq:speed} now becomes
\begin{equation}
\begin{split}
t &= d/v - d/c + t_0 \\
t &= d \times \beta^{-1} / c - d/c + t_0 \\
t &= (d / c) \times (\beta^{-1} - 1) + t_0 \\
\end{split}
\label{eq:speedred}
\end{equation}

The three assumptions relate to how the \invbeta\ and $t_0$ parameters are fixed and produce three different variables. 
The formula has two pieces of input datum, time, t, and distance, d. The distance
from the interaction point to the hit location is known to a much better degree than the time of the hit so the uncertainty on the variables comes 
almost entirely from the time measurement.

The first variable is the speed of the particle assuming it left the origin at $t_0 = 0$ reducing Eq.~\ref{eq:speed} to $t = (d / c) \times (\beta^{-1} - 1)$ or simpler
$\beta^{-1} = tc/d + 1$. The measurement of \invbeta\ comes from averaging this quantity for all the
CSC and DT hits associated with the track weighted as one over their variance.
% in the same manner as was done to caluclate their segment times.
Outlier hits from anode hits are again cleaned in the same manner as per the segment times. The weighting by one over variance and 
outlier cleaning is performed for all three measurements.

The motivation for using \invbeta\ now becomes clear, the \invbeta\ measurement
is linear with t, the source of the largest uncertainty. This means that \invbeta\ will have a much more normal shape than $\beta$ which would be skewed.
An important point here is that the distribution will be close to symmetrical for speed of light muons coming from the LHC.

The uncertainty on \invbeta\ can be calculated according to the formula
\begin{equation}
 \sigma_{1/\beta} = \sqrt{\sum_{i=1}^N \frac{(1/\beta_i - \overline{1/\beta})^2 \times w_{i}}{N-1}},
 \label{betaerr}
\end{equation}
where $\overline{1/\beta}$ is the average \invbeta\ of the track, $w_{i}$ is the weight of the ith hit, and N is the number of measurements associated with the track.

The speed of the particle is very useful in separating standard model muons from HSCP produced in new physics as is shown in Section~\ref{sec:search}.
Figure~\ref{fig:invbeta} shows the \invbeta\ measurement and its uncertainty for: data, completely dominated by collision muons; 
muons from the simulated Drell-Yan production of Z bosons and photons; cosmic ray muons, the sample is defined in~\ref{sec:search};
and HSCPs, again the sample is defined in~\ref{sec:search}. It can be seen that the data is strongly peaked at one, the cosmic ray muons are roughly flat, while
the HSCP have \invbeta\ greater than one, indicating they are traveling slowly.

\begin{figure}
  \begin{center}
      \includegraphics[width=0.44\textwidth]{figures/timing/TOF}
      \includegraphics[width=0.44\textwidth]{figures/timing/TOFErr} \\
      \caption[Distribution of \invbeta\ and the uncertainty on \invbeta]
      {Distribution of \invbeta\ and the uncertainty on \invbeta\ for data,
simulated Drell-Yan production of photons and Z boson decaying to muons (DY), muons from cosmic rays, and simulated HSCPs
        }
      \label{fig:invbeta}
  \end{center}
\end{figure}


The next variable, time at vertex, is the estimated time the particle 
left the interaction point assuming it traveled at the speed of light.
This means setting \invbeta\ to be one in~\ref{eq:speed} reducing the equation to simply $t = t_0$. 
For muons with at least a modest amount of $p_T$ that are produced in a collision in the triggered bunch crossing window this assumption is valid and thus the
value should be centered at zero. Figure~\ref{fig:vertextime} shows the time at vertex for the same three samples as in~\ref{fig:invbeta}.

\begin{figure}
  \begin{center}
      \includegraphics[width=0.44\textwidth]{figures/timing/Vertex}
      \includegraphics[width=0.44\textwidth]{figures/timing/VertexErr} \\
      \caption[Distribution of time at vertex and the uncertainty on time at vertex]
     {Distribution of time at vertex and the uncertainty on time at vertex for data,
simulated Drell-Yan production of photons and Z boson decaying to muons (DY), muons from cosmic rays, and simulated HSCPs
}
      \label{fig:vertextime}
  \end{center}
\end{figure}

The last variable, time at vertex out in, is similar but it assumes the particle is traveling into CMS, such that the parameter $t_0$ represents the time an incoming particle
would have crossed the interaction point. This can be an interesting property because tracks can be found in the inner tracker within a small time window so an incoming
cosmic reconstructed in the inner tracker would likely have a $t_0$ from this measurement near zero. 
The measurement assumes \invbeta\ = -1 reducing~\ref{eq:speed} to $t = -2 (d / c) + t_0$ which can be written to $t_0 = -2 (d / c) + t$ which makes it clear that
$t_0$ can be found as the average of this quantity with weights like the previous measurement. Figure~\ref{fig:vertexopptime} shows the distribution of this time
for the same three samples as above.

\begin{figure}
  \begin{center}
      \includegraphics[width=0.44\textwidth]{figures/timing/VertexOpp}
      \includegraphics[width=0.44\textwidth]{figures/timing/VertexOppErr} \\
      \caption[Distribution of time at vertex out in and the uncertainty on time at vertex out in]
     {Distribution of time at vertex out in and the uncertainty on time at vertex out in for data,
simulated Drell-Yan production of photons and Z boson decaying to muons (DY), muons from cosmic rays, and simulated HSCPs
}
      \label{fig:vertexopptime}
  \end{center}
\end{figure}

The question may be asked why not to make a measurement without making any assumptions on $t_0$ or \invbeta. This was checked but it was found to have
resolution worse by more than an order of magnitude and very little discrimanatory power.
This is because the assumptions in the previous measurements allowed all of them to use information
related to the beam spot, which is approximately three times as far away from the innermost part of the muon system as the outermost part is to the innermost part.
The last two measurements both assumed an error free propagation of the time in the muon system to the interaction point while the \invbeta\ measurement
added a new point at the interaction point with t = 0. This assumption free measurement is not used for any purpose in CMS.

%\section{Timing in simulation} Maybe

%\section{Conclusion}


\input{search}

\input{conclusion}

%\input{bibliography}

\newpage
\bibliographystyle{unsrt}
%\bibliographystyle{uclathes}
\bibliography{references}


\end {document}

