\chapter{Conclusion \label{sec:conclusion}}

%Our understanding of the universe on a fundamental level has evolved greatly over the last century with the discoveries of new particles and forces.
%In that time, many different theories have been put forward to explain experimental results and give predictions of future observations.
%As discussed in Ch.~\ref{sec:theory}, the current theoretical framework used is the Standard Model (SM).
%%Over the last few decades, numerous experiments have confirmed predictions made by the SM. 
%All particles predicted in the SM have been confirmed except for the Higgs Boson.
%However, in July 2012, two of the experiments at the Large Hadron Collider (LHC) observed the existence of a particle with Higgs like properties.

%As discussed in Ch.~\ref{sec:app}, the LHC accelerates protons to very high energies and then collides them with each other
%%is a 26.7 km ring located underneath the Swiss--French countryside outside of Geneva, Switzerland.
%%The LHC uses superconducting magnets to accelerate
%%protons in two counter-rotating rings to high energy, the design energy is 7 TeV but running so far has only gotten up to 4 TeV.
%%The protons are then brought to a collision 
%at a few designated points. Located at one of these points is the Compact Muon Solenoid (CMS) detector.

%CMS is a general purpose detector designed to search for a wide variety of different physics signatures.
%%with its core feature being 3.8T superconducting solenoid.
%CMS is split into numerous different subsystems, two that are of special importantance in this work are an all silicon inner tracker and an outer muon system.
%Over the last few years, CMS has collected a large amount from LHC collisions with 2012 collisions at the highest energy and rate.

%A major goal in building the LHC and CMS was the discovery of the Higgs Boson. However, they were also built in hopes of finding physics beyond the SM.
%Despite the success of the SM, there are reasons to believe it will not properly describe all physics at the high energy of the LHC.
%%These reasons include run away corrections without a large degree of fine tuning and an explanation of astronomical dark matter observed in the universe.
%Numerous theories have been put forward to explain physics beyond the SM, with one of the most popular being supersymmetry.
%In some versions of supersymmetry, such as split supersymmetry or gauge mediated supersymmetry, new heavy charged particles 
%are created with lifetimes longer than tens of nanoseconds allowing them to interact directly with the CMS detector. 
%The particles can become long-lived for different reasons, for example gluinos in split supersymmetry are long-lived

%The high mass of these new Heavy Stable Charged Particles (HSCP) would result in them traveling with speed appreciably less than the speed of light.
%That is in contrast to all the long-lived particles in the SM which would be traveling at very nearly the speed of light at the high energies of the LHC.
%HSCP can then be identified by looking for the signatures slow moving particles leave in CMS, their long time of flight to the muon system and increased
%ionization energy loss in the inner tracker.

The measurement of the arrival time of hits in the CMS muon system is important for two reasons.
The first is to quickly associate hits in the muon system with the appropriate LHC bunch crossing so that the data in the other subdetectors about this
collision can be read out. A method was developed to measure the time offset of CSC chambers with respect to a determined optimal time that would give
the best performance. 
%The method was done offline in comparison to the previous method which required dedicated running of CMS during which the data collected
%could not be used for physics analyses. 
The efficiency for chambers to correctly identify the right bunch crossing was brought to 99\%,
exceeding the design requirement of 92\%. The second reason timing is important is to help identify different types of particles.

One of the particles timing can help identify is new heavy long-lived charged particles which would live long enough to interact directly with CMS
and would be traveling at an appreciably slower speed than muons. Four searches were performed looking
for particles of this type, each being sensitive to different signatures HSCP could have inside of CMS. 
The searches were performed by looking for tracks with some combination of high momentum, late arrival time in the muon system,
or large ionization energy loss in the inner tracker.
% Specialty triggers were developed in order to collect potential signal events,
Backgrounds to the searches were muons form the collisions and in one of the searches muons from cosmic rays.
%and the prediction of their presence in the .
%The background from collision muons was predicted from data by using the lack of correlation between the selection criteria
The background in the signal region was predicted using control regions in data and the 
robustness of the prediction was checked by looking in the control region where particles would be traveling faster than the speed of light.
%The systematic uncertainty on the signal efficiency was checked by numerous studies in control regions.

No excess above the expected background from SM processes was found in any of the four searches. Limits were placed on the production of long-lived charged
particles created in many different theories, including versions of supersymmetry. These limits place important bounds on the theories
of physics beyond the SM. While no discovery of new long-lived charged particles was found in the data searched there may be other signs of
new physics present. Careful study of the recently discovered Higgs Boson-like particle is underway and many other searches are ongoing looking for the production
of new particles. In addition, after a two year shutdown, the LHC will restart operations at a higher collision energy and rate which will allowing for probing
the production of new physics at even smaller rates. 
Thus, a major discovery to reshape particle physics may be not far out of reach.
%The LHC and other experiments, present and future, will continue to explore all aspects about the building blocks of matter and how they interact,
%from the smallest scale to the largest, continually revising our understanding of nature.
%Thus, the story of particle physics still has many chapters to go.

%One of the searches 
%%consists almost entirely of my own work and 
%looks for the very unusual signature of a particle becoming charged only after traversing the inner tracker portion of CMS.
%The lack of inner tracker information leads to many difficulties not found in other CMS searches.  
%The momentum measurement, now coming from the muon system instead of the inner tracker, becomes much worse
%and the background from muons from cosmic rays becomes much larger
%To deal with the poor momentum determination a specialty trigger was developed to collect potential signal events and the collision muon background
%determination
%This was dealt with by developing a specialty trigger to collect events, developing other experimental handles, such as timing, to put less emphasis on the \pt\ measurement,
%and splitting background events by how likely 

%My work included the development and study of triggers used to collect the signal events, optimiza

%The second use is to identify different types of particles.
%% The timing of hits in the muon system is used to build higher level variables about particles.
%Muons from collisions in the LHC, muons from cosmic rays, and HSCPs would all have different timing signatures in the muon system.
%Multiple variables are built from the timing of hits in the muon system to aid in the identification of the particles, the variable used to search
%for HSCP is \invbeta.

