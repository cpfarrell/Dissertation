\chapter{Conclusion \label{sec:conclusion}}

The measurement of the arrival time of hits in the CMS muon system is important for two reasons.
The first is to quickly associate hits in the muon system with the appropriate LHC bunch crossing so that the data in the other subdetectors about this
collision can be read out. A method was developed to measure the time offset of CSC chambers with respect to a determined optimal time that would give
the best performance. 
The efficiency for chambers to correctly identify the right bunch crossing was brought to 99\%,
exceeding the design requirement of 92\%. The second reason timing is important is to help identify different types of particles.

One of the particles timing can help identify is new heavy (quasi-) stable charged particles (HSCP) which would live long enough to interact directly with CMS
and would be traveling at an appreciably slower speed than muons. Four searches were performed looking
for particles of this type, each being sensitive to different signatures inside of CMS. 
The searches were performed by looking for tracks with some combination of high momentum, late arrival time in the muon system,
or large ionization energy loss in the inner tracker.
Backgrounds to the searches were muons from the collisions and, in one of the searches, muons from cosmic rays.
The background in the signal region was predicted using control regions in data and the 
robustness of the prediction was checked by looking in the control region where particles would be traveling faster than the speed of light.

No excess above the expected background from SM processes was found in any of the four searches. Limits of around 0.01~pb were placed on the production of long-lived charged
particles created in many different theories, including versions of supersymmetry. These limits place important bounds on the theories
of physics beyond the SM. While no discovery of new long-lived charged particles was found in the data searched there may be other signs of
new physics present. Careful study of the recently discovered Higgs Boson-like particle is underway and many other searches are ongoing looking for the production
of new particles. In addition, after a two year shutdown, the LHC will restart operations at a higher collision energy and rate which will allowing for probing
the production of new physics at even smaller rates. 
Thus, a major discovery that would reshape particle physics may be not far out of reach.
