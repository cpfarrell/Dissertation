\section{Statistical Technique \label{sec:stats}}
CMS has a Statistics Committee which provides advice on statistics issues as well tools which can be used to perform statistical calculations.
Tools developed in the search for the Higgs Boson were used
in this analysis to get the significance of any observed excess and if no excess is observed to place bounds on the signal cross-section.
More information about the tools and statistical technique can be found in~\cite{Chatrchyan:2013lba}.

One of the tools calculates the significance of a signal by taking in the predicted background with its uncertainty and the observed data.
The tool calculates a test statistic, $q_0$ comparing the likelihood that observed data came from background only or signal plus background.
Then the probability of observing a test statistic at least as large as $q_0$ is found.
The smaller the probability the less likely the observed data came from background only.
The probability is returned in the form of a one-sided Gaussian sigma. For a given sigma, x, the probability p is found from Equation~\ref{eq:sigma},
\begin{equation}
p = \int_x^{\infty} \frac{1}{\sqrt{2\pi}}e^{\frac{-x^2}{2}} \mathrm{d} x.
\label{eq:sigma}
\end{equation}
the function being integrated over is a normalized gaussian with unit variance. Particle physics has a convention that one claims a discovery when the significance
is greater than five sigma and evidence when it is greater than three. A five sigma discovers means that there is a one in 3.5 million probability that
observed data come from background only.

This tool can also be used find the expected reach of the analysis. The expected reach of the analysis is defined as the signal cross-section
for which there is a 50\% chance of being able to claim a discovery.
This is of particular interest when designing the analysis to determine optimum thresholds on the selection variables.

Another tool returns the expected and observed limits on the signal cross-section. The tool is passed the different sources of background with their uncertainties,
the signal efficiency with its uncertainty, and finally the integrated luminosity with its uncertainty. The tool then proceeds to calculate the cross section limits with
a hybrid CLs approach~\cite{Read:451614} using a profile likelihood technique~\cite{Cowan:2010js} with the predicted background, signal efficieny, and integrated luminosity
as nuisance parameters using lognormal pdfs~\cite{Eadie,James}. The uncertainty on the predicted background is discussed in Section~\ref{BackPred} and the
signal efficiency uncertainty is discussed in Section~\ref{sec:SystUnc}. The uncertainty on integrated luminosity is 4.4\%~\cite{SMP-12-008}.