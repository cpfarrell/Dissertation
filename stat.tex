\section{Statistical Technique \label{sec:stats}}
CMS has a Statistics Committee which provides advice on statistics issues as well tools through the RooStats package~\cite{Moneta:2010pm}
which can be used to perform statistical calculations.
Tools developed in the recent CMS search for the Higgs Boson were used
in this analysis to determine the significance of any observed excess and, if no excess is observed, to place bounds on the signal cross-section.
More information about the tools and statistical technique can be found in~\cite{Moneta:2010pm, Cowan:2010js, Chatrchyan:2013lba}.

In particle physics, the probability $p$ that the observed data come from background processes only is generally expressed
as a significance using a one-sided Gaussian convention. Specifically, the number of sigma $s$ is found by satisfying the relation
\begin{equation}
p = \int_s^{\infty} \frac{1}{\sqrt{2\pi}}e^{\frac{-x^2}{2}} \mathrm{d} x.
\label{eq:sigma}
\end{equation}
where the function being integrated over is a normalized gaussian with unit variance. Particle physics has a somewhat arbitrary
convention that one claims a discovery when the significance
is greater than five sigma, and evidence when it is greater than three. A five sigma discovery is equivalent to there being a one in 3.5 million chance that
observed data come from background only.

One of the tools calculates the significance of a signal by taking in the predicted background with its uncertainty and the observed data.
The tool calculates a test statistic $q_0$ which is equal to 
\begin{equation}
-2 \ln \frac{L(obs|b)}{L(obs|s+b)}
\label{eq:q0}
\end{equation}
where $L(obs|b)$ and $L(obs|s+b)$ are the likelihoods that the observed data came from background only or signal and background, respectively.
The $s$ in $L(obs|s+b)$ is found by maximizing the likelihood function. 
Then the probability of observing a test statistic at least as large as $q_0$ is found.
The smaller the probability the less likely the observed data came from background only.

This tool can also be used to find the expected reach of the analysis by finding the expected significance for a given signal cross-section.
The expected reach of the analysis is defined as the signal cross-section for which there is a 50\% chance of being able to claim a discovery.
This is of particular interest when designing the analysis to determine optimum thresholds on the selection variables.

Another tool returns the expected and observed limits on the signal cross-section. The tool is passed the different sources of background with their uncertainties
(both statistical and systematic), the signal efficiency with its uncertainty (both statistical and systematic),
and finally the integrated luminosity with its uncertainty (systematic only). The tool then proceeds to calculate the cross section limits with
a hybrid $CL_s$ approach~\cite{Read:451614} using a profile likelihood technique~\cite{Cowan:2010js} with the predicted background, signal efficiency, and integrated luminosity
as nuisance parameters using lognormal pdfs~\cite{Eadie,James}. The uncertainty on the predicted background was discussed in Section~\ref{BackPred} and the
signal efficiency uncertainty is discussed in Section~\ref{sec:SystUnc}. The uncertainty on integrated luminosity is 4.4\%~\cite{SMP-12-008}.