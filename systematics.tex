\section{Signal Efficiency Systematic Uncertainties \label{sec:SystUnc}}

The signal efficiency for all the analyses is determined from MC. To assess how well MC matches data, numerous studies were performed using control samples.

\subsection{Trigger Efficiency Uncertainty}


The uncertainty on the muon trigger efficiency can come from numerous different effects. 
A 5\% difference in the muon trigger efficiency has been observed between data and MC~\cite{2012JInst...7P0002T}.
An additional uncertainty important to slow moving particles is the timing synchronization in the muon system. As an HSCP arrives in the muon system
closer to the switching of the assigned bunch crossing window, a discrepancy in the modeling of the timing synchronization in MC would have a larger effect than for SM particles.
The timing of the trigger system is set at the local system level with the
exact segmentation differing between the three muon subsystems.
%For the CSCs, the timing is set by the chamber that the particle passes through.
%The average trigger timing  with respect to the LHC clock of each of the trigger timing components for each muon subsystem was measured from data, for the CSCs this means
%a relative timing was found for each chamber.

To determine the size of the effect, the average and RMS of the synchronization of the trigger timing components with respect to the LHC clock
for each subsystem was measured.
A normalized gaussian distribution was then created for each subsystem with the mean and width equal to the measured values.
Then, each trigger timing component was
assigned a shift value drawn from the gaussian representing the muon subsystem to which it belongs. The simulation of the detector electronics was then repeated
for signal MC samples, with
the time of the simulated hits in the detector shifted by the value associated with the portion of the detector the hit is in. Then the reconstruction and trigger
simulation steps were redone. 
%The effect on the efficiency for each of the triggers used in the analyses plus the logical OR of the triggers is shown in
%Fig.~\ref{fig:MuSynch} (left) for the samples expected to display the largest effect.
%From the rightmost bin,
Multiple samples were evaluated including the highest mass samples which would have the largest effect.
The largest efficiency change was 4\% and this is taken as the uncertainty for all samples that use the muon trigger.

%\begin{figure}
%\centering
%  \includegraphics[clip=true, trim=0.0cm 0cm 2.8cm 0cm, width=0.44\textwidth]{figures/syst/syst_8TeVMatchedSAEffLost}
%  \includegraphics[clip=true, trim=0.0cm 0cm 0cm 0.0cm, width=0.44\textwidth]{figures/syst/summaryEffLossMET55}
%\caption{Relative change in trigger efficiency seen after applying uncertainties for various signals.
%Left: Shifting the synchronization of the muon trigger system as a function of different triggers. The last bin on the X-axis is for the logical OR of the three triggers.
%Right: Adjusting the jet energy scale and recalculating MET. From left to right bins have only JEC applied, only JEU applied, and both JEC and JEU applied.
%    \label{fig:MuSynch}}
%\end{figure}

%Fractionally charged particles have additional systematic uncertainty because their smaller energy depositions mean their hits may not be above thresholds in the muon system
%and tracks will not be found. This simulation of muon system electronics and gas gain can affect this reconstruction efficiency. This was modelled by shifting
%the gain in the muon system by 25\%~\cite{GasGain}. This variation resulted in a signal efficiency change of 15\% (3\%) for $Q = e/3$ ($2e/3$).

Also contributing to the trigger uncertainty is the accuracy of PFMET in the trigger in MC. The uncertainty on the PFMET is dominated by the uncertainty on
the measurement of jet energies. A group within CMS studies the agreement between data and MC for jets. For both data and MC,
the group releases corrections to the energy scale of jets (JEC) which can be applied after reconstruction to give the best measurement of the energy of the jet. The corrections
come with corresponding uncertainties (JEU). The jets used for calculating the PFMET at trigger level are not corrected.

To determine the systematic uncertainty, the jets in the signal MC are adjusted by both the JEC and JEU and the MET is recalculated.
The jets are corrected by the MC JEC and then by the inverse of the data JEC.
This results in the MC jets having the same properties as uncorrected jets in data.
The JEU are applied by decreasing the energy of each jet by its uncertainty.
%Figure~\ref{fig:MuSynch} shows the efficiency change when applying the JEC and JEU, both individually and together. 
The JEC are found to increase the efficiency slightly while the JEU decrease it by approximately 1\%.
Conservatively, no scale factor is applied on the trigger efficiency and a 1\% uncertainty is taken on the MET trigger for all samples.

The total systematic uncertainty on trigger efficiency is found by combining the above effects. 
The muon trigger is dominant for all samples except for the charge suppressed samples where the muon trigger does not have any efficiency.
For those samples, the trigger efficiency uncertainty comes only from the MET uncertainty.

%The evaluation of trigger uncertainties requires the creation of new samples which take a large
%amount of time and computing resources to create, making it not feasible to evaluate them for all mass and model points, individually. Instead, the uncertainties were evaluated
%for a few representative points including the samples expected to have the largest effect. Then the remaining signal points were assigned one of the evaluated
%uncertainties, always being conservative and assigning the uncertainty from a signal point which would have a larger effect.

\subsection{Uncertainty on Selection Variables}

The uncertainty on the \invbeta\ measurement is studied using muons from the decay of Z bosons. Muons are required to pass a tight selection provided by the muon POG to
give a pure muon sample. Additionally the event must have a pair of oppositely charged muons with an invariant mass of $M_Z \pm 10$~GeV. Only the two muons forming the combination
are used. If more than one such pair exists, the pair with invariant mass closest to $M_Z$ is used. An uncertainty of 0.005 is taken on the \invbeta\ measurement
from the disagreement between data and MC. This uncertainty is evaluated separately for all samples and is found
to be less than 7\% for all considered signals.

The uncertainty due to the $p_T$ measurement is determined by varying the $1/p_T$ value by a prescription from the muon POG~\cite{2012JInst...7P0002T}. For the \muononly\
analysis the $1/p_T$ of the muon system track is shifted up by 10\%, this means that the \pt\ will decrease.
For the other analyses the $1/p_T$ of the inner track is adjusted by the Equation

\begin{equation}
 \frac{1}{p_T\prime} = \frac{1}{p_T} + \delta_{K_T}(q, \phi, \eta)
\end{equation}

\begin{equation}
 \delta_{K_T}(q, \phi, \eta) = A + B\eta^2 + qC\sin(\phi - \phi_0)
\end{equation}
where A = 0.236 TeV$^{-1}$, B = -0.135 TeV$^{-1}$,
C = 0.282 TeV$^{-1}$, and $\phi_0$ = 1.337. The shifts were found to have a less than 10\% effect on the efficiency to pass the final selection for all signals.

The effect of the uncertainty on \dedx\ was evaluated with low momentum protons. Protons with $p$ less than  about 2~GeV will have speed appreciably lower than the
speed of light and thus will appear similar to signal candidates. A comparison of data and MC yields an uncertainty of 0.05 on \ias\ and 5\% on \ih. %These uncertainties,
When propagated to the final selection for singly charged particles, these uncertainties give efficiency changes of less than 13\% for low mass samples
and less than 7\% for masses above 200~GeV/$c^2$. Multiply charged particles have sufficient
separation between signal and background that the uncertainty is negligible.

%For fractionally charged particles the uncertainty on \dedx\ also affects the track reconstruction efficiency as if the energy is too low the hits will not be reconstructed.
%To evaluate this 

\subsection{Other Uncertainties on Signal Efficiency}

The systematic uncertainty on muon~\cite{2012JInst...7P0002T} and track~\cite{CMS-PAS-TRK-10-002} reconstruction were both found to be less than 2\%.
A 1\% uncertainty recommended by the muon POG is applied on the correction factors used in the \muononly\ analysis described in Section~\ref{sec:TagProbe}.

The uncertainty on the number of proton-proton collisions per bunch crossing is found by varying by 6\% the 
proton-proton cross-section used to determine the weights for MC events (see end of Sec.~\ref{sec:samples}).
This leads to an uncertainty of less than 4\% for all samples.

Multiply charged particles deposit a large amount of energy in the calorimeter and if the particle does not have enough energy it may come to a stop in the calorimeter.
This is particularly important for high charge, low mass samples, because if the particle has enough energy to pass through the calorimeter then it will
be traveling at a high speed and be unlikely to pass the threshold on \invbeta. Uncertainty on the amount of detector material, determined mostly by the
HCAL, affects the number of HSCP that will stop. There is a conservative 5\% uncertainty on the material budget as measured by energy loss of cosmic ray muons
passing through CMS~\cite{2010JInst...5T3021C}. Shifting the material density by this amount for a few signal samples has at most a 20\% effect. This 20\% uncertainty is
conservatively taken for all multiply charged samples.

\subsection{Total Signal Efficiency Uncertainty}

The total systematic uncertainty for each signal point is found by adding the above effects in quadrature.
Figure~\ref{fig:MuOnlyUncSource} shows the different sources of signal efficiency systematic uncertainty and their quadratic sum
for the various signal models considered in the \muononly\ analysis.
Figure~\ref{fig:TkMuStauUncSource} and~\ref{fig:TkMuRHadUncSource} shows the same for the \tktof\ analysis for stau and $R-hadron$ models, respectively.
Figure~\ref{fig:TkOnMCUncSource} shows the sources of systematic uncertainty for a few of the samples in the \tkonly\ and \multi\ analyses.

\begin{figure}[ht]
\centering
  \includegraphics[clip=true, trim=0.0cm 0cm 2.8cm 0cm, width=0.44\textwidth]{figures/muonly/MoGluino_f100Uncertainty}
  \includegraphics[clip=true, trim=0.0cm 0cm 2.8cm 0cm, width=0.44\textwidth]{figures/muonly/MoGluino_f50Uncertainty} \\
  \includegraphics[clip=true, trim=0.0cm 0cm 2.8cm 0cm, width=0.44\textwidth]{figures/muonly/MoGluino_f10Uncertainty}
  \includegraphics[clip=true, trim=0.0cm 0cm 2.8cm 0cm, width=0.44\textwidth]{figures/muonly/MoStopUncertainty}
\caption[Relative signal efficiency change seen for the various sources of uncertainty in the \muononly\ analysis]
{Relative signal efficiency change seen for the various sources of uncertainty in the \muononly\ analysis.
Top row: Gluino with $f=1.0$ (left) and $f=0.5$ (right).
Bottom row: Gluino with $f=0.1$ (left) and stop (right)}
    \label{fig:MuOnlyUncSource}
\end{figure}

\begin{figure}[ht]
\centering
  \includegraphics[clip=true, trim=0.0cm 0cm 2.8cm 0cm, width=0.44\textwidth]{figures/tkmu/MuGMStauUncertainty}
  \includegraphics[clip=true, trim=0.0cm 0cm 2.8cm 0cm, width=0.44\textwidth]{figures/tkmu/MuPPStauUncertainty} \\
\caption[Relative signal efficiency change seen for the various sources of uncertainty for stau models in the \tktof\ analysis]
{Relative signal efficiency change seen for the various sources of uncertainty for stau models in the \tktof\ analysis.
CD (left) and DP (right) models.}
    \label{fig:TkMuStauUncSource}
\end{figure}

\begin{figure}[ht]
\centering
  \includegraphics[clip=true, trim=0.0cm 0cm 2.8cm 0cm, width=0.44\textwidth]{figures/tkmu/MuGluino_f50Uncertainty}
  \includegraphics[clip=true, trim=0.0cm 0cm 2.8cm 0cm, width=0.44\textwidth]{figures/tkmu/MuGluino_f10Uncertainty}\\
  \includegraphics[clip=true, trim=0.0cm 0cm 2.8cm 0cm, width=0.44\textwidth]{figures/tkmu/MuStopUncertainty}
\caption[Relative efficiency change seen for the various sources of uncertainty for $R-hadron$ models in the \tktof\ analysis]
{Relative efficiency change seen for the various sources of uncertainty for $R-hadron$ models in the \tktof\ analysis.
Top row: Gluino with $f=0.5$ (left) and $f=0.1$ (right).
Bottom row: Stop}
    \label{fig:TkMuRHadUncSource}
\end{figure}

\begin{figure}[ht]
\centering
  \includegraphics[clip=true, trim=0.0cm 0cm 2.8cm 0cm, width=0.44\textwidth]{figures/tkonly/TkGluinoN_f10Uncertainty}
  \includegraphics[clip=true, trim=0.0cm 0cm 2.8cm 0cm, width=0.44\textwidth]{figures/tkonly/TkStopNUncertainty}\\
  \includegraphics[clip=true, trim=0.0cm 0cm 2.8cm 0cm, width=0.44\textwidth]{figures/multi/HQDY_Q4Uncertainty}
  \includegraphics[clip=true, trim=0.0cm 0cm 2.8cm 0cm, width=0.44\textwidth]{figures/multi/HQDY_Q7Uncertainty}\\
\caption[Relative signal efficiency change seen for the various sources of uncertainty for some of the models considered in the \tkonly\ and \tktof\ analyses]
{Relative signal efficiency change seen for the various sources of uncertainty.
Top row: Charge suppressed Gluino with $f=0.1$ (left) and charge suppressed stop (right) in the \tkonly\ analysis.
Bottom row: Q = 4e (left) and 7e (right) in the \multi\ analysis.}
    \label{fig:TkOnMCUncSource}
\end{figure}

The total signal efficiency uncertainty for all considered models is shown in
Figure~\ref{fig:TotalUnc} for the four analyses. The signal efficiency uncertainty used for each signal point is what is shown in this figure.
For all analyses except for the \multi\ analysis, the uncertainty is less than 15\% for all signal points and less than 10\% for a large majority of signal points.
The large uncertainty on the amount of detector material results in the uncertainty for the multiply charged particles to be between 20\% and 30\%.

\begin{figure}[ht]
\centering
  \includegraphics[clip=true, trim=0.0cm 0cm 2.8cm 0cm, width=0.44\textwidth]{figures/muonly/MOUncertainty}
  \includegraphics[clip=true, trim=0.0cm 0cm 2.8cm 0cm, width=0.44\textwidth]{figures/tkmu/MuUncertainty}
  \includegraphics[clip=true, trim=0.0cm 0cm 2.8cm 0cm, width=0.44\textwidth]{figures/tkonly/TkUncertainty}
  \includegraphics[clip=true, trim=0.0cm 0cm 2.8cm 0cm, width=0.44\textwidth]{figures/multi/HQUncertainty}
\caption[Total signal efficiency uncertainty for all considered models]
{Total signal efficiency uncertainty for all considered models.
Top:  For the \muononly\ (left) and \tktof\ (right) analyses.
Bottom:  For the \tkonly\ (left) and \multi\ (right) analyses.}
    \label{fig:TotalUnc}
\end{figure}

