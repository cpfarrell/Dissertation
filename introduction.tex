\chapter{Introduction}

%Throughout human history there has been a desire to understand nature at its most fundamental level.
%Particle physics is concerned with studying the nature of fundamental particles and the interactions between them. 
Particle physics is concerned with studying what matter is made of and how it interacts at a fundamental level.
%he matter could be everyday objects or far away stars.
The desire to understand nature at its most basic level has long been of interest to humankind, dating back to at least the ancient Greek philosophers.
The field of particle physics began to come into its own beginning with the discovery of the electron in 1897 by J.J. Thompson~\cite{griffiths2008introduction}.
In the more than a century since that discovery, the field has been revolutionized many times. Whenever it seemed that the final piece of the puzzle was within reach
a surprising result would be found, fundamental particles shown to be composites, inviolate symmetries found to be broken,
or an expanding universe in contradiction to gravitational expectation.

The current state of knowledge in particle physics is contained in the standard model (SM) of particle physics. The SM contains all of the known
particles and the way that they interact with one another. The SM has been developed over the last forty years and has proven to be a very successful theory. 
Numerous experiments have validated the theory with the discovery of predicted particles or agreement of parameter values.

The last particle predicted by the SM to be found is the Higgs Boson.
% predicted in 1974 (cite Higgs theory).
The discovery of the Higgs Boson is one of the reasons the Large Hadron Collider (LHC) was built outside of Geneva, Switzerland. 
The LHC collides protons at an energy higher than any previous experiment at an extremely high rate~\cite{1748-0221-3-08-S08001}
producing many particles in the collisions.
% including Higgs Bosons if they exist. 
Surrounding the points where the LHC brings the protons to a collision
are experiments designed to measure and reconstruct the particles emanating from the collisions. Two of these experiments, CMS and
ATLAS, announced in June of 2012 the discovery of a new particle with properties like that of the SM Higgs Boson~\cite{Chatrchyan:2013lba, Aad:2012tfa}, possibly
indicating that the last piece of the SM has been found.

%However it may be that the discovered particle is not the SM Higgs but comes from another source (cite papers saying not Higgs).
Following the history of particle physics, just when the theory looks to be the most stable can be when a discovery revolutionizes the field.
There are many models which predict physics beyond the SM~\cite{Martin:1997ns, Tata:1997uf} 
that would be produced by the LHC and could be detected by the LHC experiments.
One of the most popular models is supersymmetry (SUSY) where all of the SM particles are given a superpartner with spin different by one half.
Some of these models, including versions of SUSY, predict the production of new heavy (meta-)stable charged particles (HSCP) 
which would directly interact with the LHC experiments. All of the long-lived particles in the SM have a small mass meaning that a discovery
would be a clear indication that a new theory must be developed to explain these particles.

There are numerous different types of HSCP predicted in the models of new physics and the different types can leave exotic signatures in the LHC detectors.
Some types of HSCPs would form composite objects with SM particles and undergo nuclear interactions with the detector material
that change the SM particles in the composite particles. This could lead to the HSCP changing its electric charge during its propagation through the detector.
Other HSCP could be produced with electric charge not equal to e, the charge of an electron, unlike almost all electrically charged SM particles.

Multiple complimentary searches for HSCP produced at the LHC were carried out using data collected by the CMS experiment and are presented in this work.
Each of the searches is designed to be sensitive to different signatures that HSCP could have in CMS but many of the tools and techniques used are
common between searches. The searches exploit the characteristics of HSCP that allow them to be separated from the very large background of SM particles.
Requirements are placed on these characteristics and the residual background due to SM particles is evaluated. The data are then checked to see whether
the observation is consistent with this background. If it is not consistent, then this indicates the presence of a new particle beyond the SM while if it
is consistent then this places limits on models of physics beyond the SM.

One of the characteristics that separate an HSCP from a SM particle is the late arrival of HSCP to the outer portion of CMS due to its large mass making it slow moving.
In order to observe this, it must be possible to measure the arrival time of hits in the outer portions of CMS. This time measurement is important not only
for HSCP but also for SM particles in order to associate them to the correct LHC beam collision and to separate out SM particles not coming from
LHC collisions, such as from cosmic rays.

Chapter~\ref{sec:theory} of this work discusses the SM and a few models of physics beyond the SM. 
An emphasis is placed on theories which include new long-lived charged particles.
%the theoretical basis of particle physics is discussed. This includes the
%current theoretical framework that is used to describe the known particles and forces, the Standard Model (SM).
%Additionally, theories of possible physics beyond
%the SM are considered. An emphasis is placed on theories which include new long-lived charged particles as 
%a search for these particles form the basis of this work.
In Chapter~\ref{sec:app}, the experimental apparatus used in this work is presented, this includes the LHC and 
CMS, in particular parts of the apparatus especially important
for searches for long-lived charged particles are given extra detail.
Chapter~\ref{sec:timing} discusses the measurement of the arrival time of hits in the outer portion of CMS.
%The timing performance is shown in both the low latency online environment and the offline determination of the timing of a particle.
Chapter~\ref{sec:search} details four complimentary searches for new heavy long-lived charged particles.
%using data collected with CMS from collisions at the LHC.
%The different searches focus on different signatures new particles could leave in CMS depending their properties and how they interact with CMS
%The searches were performed with
%the CMS collaboration and are in the process of being submitted to a journal, along with one more search. One of the searches
%is based almost entirely on my own work while I was one of the lead analyzers in a small group for the other three.
Concluding remarks on the presented work are given in Chapter~\ref{sec:conclusion}.