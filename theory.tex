\chapter{Relevant Theoretical Issues \label{sec:theory}}

\section{Introduction}
Our understanding of particles and how they interact is constantly evolving through new experimental results and theoretical breakthroughs.
In this chapter, the current theoretical framework in particle physics is briefly described. Also, theories of physics beyond this framework are introduced,
with a focus on theories which predict the existence of new long-lived charged particles.

\section{Standard Model \label{sec:SM}}
The Standard Model (SM) of particle physics is a framework for describing fundamental and composite particles and the forces that govern how they interact. 
More information about the Standard Model can be found in~\cite{Srednicki_2007, griffiths2008introduction}

Particles in the SM are split into bosons, which have integer spin, and fermions, which have half integer spin.
The bosons are the carriers of the forces of the SM while the fermions act as the matter fields.

%The forces in the SM are the electromagnetic, strong, and weak forces. The carrier of the strong force is the 

Formally, the SM is described according to the symmetry group
\begin{equation}
SU(3)_C \times SU(2)_L \times U(1)_Y
\label{eq:SMGroups}
\end{equation}
where SU(N) denotes special unitary groups of dimension N. 
%Special unitary groups have the property of having a determinant of one.
Reading from left to right the groups represent color, weak isospin, and hypercharge. 
The color symmetry group is responsible for the strong force which is carried by the gluon.
A mixture of the weak isospin and hypercharge groups are responsible for the weak and electromagnetic forces. Left handed fermions are doublets of the
isospin group while the right handed fermions are singlets as they do not interact with the raising and lowering operators of the SU(2) group.
The weak force is carried by the W and Z bosons while the electromagnetic force is carried by the photon.

The last of the bosons in the SM is the Higgs Boson. The Higgs Boson is a scalar which allows it to have a non-zero vacuum expectation value (VEV).
This non-zero VEV breaks the electroweak symmetry and results in the W and Z bosons acquiring mass while the photon remains massless.
The Higgs Boson's non-zero VEV also gives mass to the fermions by allowing for a coupling between the singlet right handed fermions and doublet left handed fermions
which is otherwise not allowed.
%Describe the Higgs mechanism in this paragraph, don't really know how to do it currently.
In June 2012, the CMS and ATLAS experiments at the Large Hadron Collider
(see Ch.~\ref{sec:app}) announced the discovery of a new particle with properties similar to the Higgs Boson. If the particle is confirmed
to be the Higgs Boson, this would represent the final particle of the SM to be discovered.

The fermions in the standard model are split between quarks and leptons. Both the quarks and leptons are arranged into three families
with each family containing two quarks and two leptons. 
%Figure~\ref{fig:SM} (source CERN) shows all the confirmed particles in the SM along with their mass, electric charge, and spin.
The first family contains the up and down quarks, the electron, and the electron neutrino.
The second family has the strange and charm quarks, the muon, and the muon neutrino. The third family contains the bottom and top quarks,
the tau, and the tau neutrino. The down, strange, and bottom quarks have charge -1e/3 while the up, charm, and top quarks have charge +2e/3.
The electron, muon, and tau have charge -1e and all of the neutrinos are electrically neutral. All of the electrically charged particles in the SM have
an antiparticle with the opposite charge.

%\begin{figure}
%  \begin{center}
%      \includegraphics[clip=true, trim=0.0cm 0cm 0.0cm 0cm, width=1.0\textwidth]{figures/SM}\\
%      {Confirmed particles in the SM. Source: CERN
%        }
%      \label{fig:SM}
%  \end{center}
%\end{figure}

The quarks and gluons have color charge and interact through the strong force. There are three copies of each quark for each of the three different color charges.
The strength of the strong force increases with distance leading to color confinement where no free particles can exist with color charge.
Thus all quarks and gluons will form composite particles that are color neutral, called hadrons. There are three combinations of particles that lead to color
neutral composites. The first is a quark and an anti-quark referred to as mesons such as a charged pion which is made of an up quark and and down antiquark.
The second is three quarks (or anti-quarks) referred to as baryons and includes the proton and neutron.
Mesons and baryons can be either electrically charged or neutral.
The third is a pair of gluons, this type of particle is theoretically allowed but has never been experimentally observed. It would always be electrically neutral.

When quarks and gluons are produced at high energies, the binding of the strong force normally results in numerous secondary quarks and antiquarks being produced.
These quarks and antiquarks will then form hadrons in a process called hadronization.
%form composite particles after being created at particle colliders like the LHC, they will normally create a large number of particles, called hadronization. 
All of the hadrons will be traveling in roughly the same direction resulting in a stream of colinear particles from the production point.
%The result is a large number of nearly colinear particles emanating from the interaction point. 
This beam of partilces is referred to as a jet.

Most particles in the SM, both fundamental and composite, have very short lifetimes preventing them from being directly detected. The only stable particles
in the SM are the electron, proton, photon, and the neutrinos. Additionally, a free neutron has a lifetime of eight minutes but it may be stable inside a nucleus.
A few other particles have lifetimes long enough to be detected before decaying including the muon, pion, and kaon.

The proton-proton collisions at the LHC will create vast numbers of these SM particles. As the strong force has the largest coupling,
a large majority of the events will be the production of light quarks and gluons. A small fraction of the events, though a large total number given the very
high rate of collisions at the LHC, will produce particles such as W and Z bosons, top quarks, and possibly, if it is confirmed, the Higgs Boson.
These particles will quickly decay into other SM particles like muons, electrons, and b quarks. This production of stable and long lived SM particles,
particularly muons, form part of the background for the search for HSCP detailed in Chapter~\ref{sec:search}.

%\subsection{Cosmic rays \label{sec:cosmics}}
The production of SM particles proceeds not only in the proton-proton collisions at the LHC but also through astrophysical processes.
The earth is constantly being bombarded with high momentum protons from astronomical sources. These protons interact with the earth's atmosphere
resulting in the production of numerous charged and neutral pions. Charged pions will then decay into high momentum muons.
As the muons will have a large relativistic boost, they will be able to reach the earth before decaying.
%At sea level the flux of muons is approximately one per 10$cm^2$.
Muons lose only a small amount of energy in interactions with matter,
allowing the highest energy cosmic ray muons to penetrate through large amounts of earth.
This allows cosmic ray muons to reach the detectors of the LHC potentially creating a background for searches for new physics.

\section{Beyond Standard Model Theories and Heavy Stable Charged Particles \label{sec:BSM}}
More information about theories beyond the SM and heavy stable charged particles can be found in~\cite{Fairbairn:2006gg, Martin:1997ns, Tata:1997uf}.

While the SM has proven to be a very robust theory, there are numerous reasons to believe it is not complete. The reasons for this include runaway
radiative corrections and large amounts of fine tuning of parameters. At a minimum, the theory must be replaced at the Planck energy scale ($10^{18} GeV$)
where the gravitational force becomes as strong as the other forces. To address issues like these numerous
theories have been put forth for physics beyond the SM (BSM). 
If these BSM theories are accurate, evidence of them would likely be present in the high energy collisions produced at the LHC.
Some of these BSM theories predict the existence of heavy meta-stable
charged particles (HSCP) with lifetimes greater than a few ns, long enough to traverse the length of typical particle detectors. 

One of the most popular BSM theories is supersymmetry (SUSY). In SUSY, a new symmetry is added to the SM which gives
each SM particle a superpartner particle with spin different by one half. 
The names of the SUSY particles are generally found by prepending an s (for scalar) to the SUSY partners of spin 1/2 particles,
so the SUSY partner of the electron is the selectron, and adding ino to the end of other particles, so the higgs SUSY partner is the higgsino.
As no SUSY particles have yet been discovered the
symmetry must be broken at some scale giving the SUSY particles masses larger than SM particles. In order to address the unresolved issues in the SM, this mass gap is
expected to be no larger than about 1 TeV. In addition to adding a new symmetry, SUSY also predicts the existence of a new multiplicatively conserved quantity called R-parity.
SUSY particles have an R-parity value of -1 while SM particles have a value of 1. This implies that the lightest SUSY particle (LSP) will be stable and in most
SUSY theories it is taken to be electrically and color neutral so as to be the astronomically observed dark matter.

Other SUSY particles besides the LSP could have a long lifetime in certain areas of SUSY parameter space. In the minimal supersymmetric standard model (MSSM) the LSP is the
neutralino (superpartrner of a neutrino) in most cases. The next lightest SUSY particle (NLSP) can be long-lived if the mass splitting between the NLSP and the LSP is small, this
can happen for many different particles as the NLSP. 
Non-universal squark masses can be used to make the mass difference between the stop and the neutralino too small for the stop decay to a neutralino and
a bottom quark to be kinematically allowed.
%One case of interest is that of the scalar top (stop $\tilde{t}$) as the NLSP motivated by electroweak
%baryogenesis~\cite{Balazs:2004bu}. If the mass difference between the stop and the neutralino makes the stop decay to a neutralino and a b quark kinematically not
%allowed as arranged by non-universal squark masses, 
Then the stop decay happens via the radiative decay to a charm quark and neutralino making the stop very long-lived.

Another variant of supersymmetry is split SUSY. In split SUSY, scalar SUSY particles have very large masses while other particles remain at the TeV scale.
As gluinos $\tilde{g}$ (superpartners of the gluon) must decay through squarks this can make the gluino quite long-lived.

Gluinos and stops have color charge and as such will form composite hadrons with SM quarks and gluons after production, referred to as $R-hadrons$.
These $R-hadrons$ can be mesons, baryons, or, for gluinos, a glue-ball made of a gluino and a SM gluon.
%Examples of R-hadrons are shown in Fig.~\ref{fig:Composites}.
$R-hadrons$ can be electrically neutral or have charge Q, taken here and everywhere else in this paper unless otherwise stated as the absolute value of the charge,
of 1e or 2e, where e is the charge of the electron.
One particularly interesting case is for glue-balls which will always be electrically neutral. The fraction of gluinos forming glue-balls is
a free parameter in the theory. If the fraction is 100\% then all gluino $R-hadrons$ will be produced electrically neutral.

%\begin{figure}
%  \begin{center}
%      \includegraphics[clip=true, trim=0.0cm 0cm 0.0cm 0cm, width=1.0\textwidth]{figures/quarks}\\
%      {Different types of R-hadrons that color charged HSCP can form. Top left: Stop HSCP baryon with two SM quarks.
%Top right: Stop HSCP meson with a SM anti-quark. Bottom left: Gluino HSCP baryon with three SM quarks. Bottom right: Gluino HSCP glue-ball with a SM gluon.
%	}
%      \label{fig:Composites}
%  \end{center}
%\end{figure}

After the $R-hadrons$ are produced at the LHC, they will propagate out to and interact with the LHC detectors. In the interactions with the
detectors it is possible for the electrical charge of the $R-hadron$ to change, possibly going from neutral
to charged or from charged to neutral. The process occurs through an exchange of quarks with the detector material in nuclear interactions.
%an example of this can be seen in Fig~\ref{fig:Rhadron} taken from ref.~\cite{SMP}.
The modelling of these interactions has some uncertainty, of particular interest is the fraction of $R-hadrons$ that will be electrically charged after an interaction.
Two models are considered in this work.
The first is the model presented in~\cite{Kraan:2004tz, Mackeprang:2006gx} 
which is referred to as the cloud model where the $R-hadron$ is pictured as a spectator HSCP surrounded by a cloud of light, color-charged SM quarks and gluons.
This model results in a mixture of neutral and charged $R-hadrons$ after a nuclear interaction.
The second model, referred to as charge-suppressed, results in all $R-hadrons$ becoming neutral after a nuclear interaction as described in~\cite{Mackeprang:2009ad}.
%Most HSCP will not have a nuclear interaction while passing through the CMS tracker however a very large majority will have one in the calorimeter system.

%\begin{figure}
%  \begin{center}
%      \includegraphics[clip=true, trim=0.0cm 0cm 0.0cm 0cm, width=1.0\textwidth]{figures/Rhadron}\\
%      {R-hadron interaction
%	}
%      \label{fig:Rhadron}
%  \end{center}
%\end{figure}

A third possible HSCP in SUSY  is the production of long-lived staus $\tilde{\tau}$ in gauge mediated symmetry breaking (GMSB) SUSY~\cite{Giudice:1998bp}. 
In GMSB, the gravitino is very light and almost always the LSP.
GMSB models are characterized by six parameters which determine the mass heirarchy and decays of SUSY particles.
One of these parameters is the number N of SU(5) chiral multiplets added to the model which act as ``messengers''.
As long as N is not too small the NLSP is likely to be the stau.

The lifetime of the stau is given by~\cite{Fairbairn:2006gg}
\begin{equation}
\tau_{Stau} = 0.1 \left(\frac{100 GeV}{m_{Stau}}\right)^5 \left(\frac{m_{\tilde{G}}}{2.4 eV}\right) mm/c
\label{eq:lifetime}
\end{equation}
with $m_{\tilde{G}}$, the gravitino mass, set by
\begin{equation}
m_{\tilde{G}} = 2.4 c_{Grav} \left(\frac{\sqrt{M\Lambda}}{100 TeV}\right)^2 eV
\label{eq:gravmass}
\end{equation}
with $c_{Grav}$ being the ratio between the fundamental SUSY breaking scale and the effective one felt by the messenger particles,
$\Lambda$ the effective SUSY breaking scale, and M the mass of the messengers.
The parameter $C_{Grav}$ relates to how the SUSY breaking is transmitted to the messengers, if the communication is done perturbatively then $c_{Grav}$
will be very large giving the stau a long lifetime.

Other BSM theories besides SUSY can also contain HSCPs.
An interesting scenario for HSCP is the production of particles with charge not equal to $1e$.
One model that includes non-unit charged HSCP is the production of particles that are neutral under $SU(3)_C$ and $SU(2)_L$ 
but have electric charge meaning they only couple to the photon and Z boson through $U(1)_Y$ interactions~\cite{Langacker:2011db}.
%The electric charge Q of the particle is not constrained to be 1e.
%Note here and for the rest of this paper Q is taken as the absolute value of the electrical charge unless specifically stated otherwise.
The HSCP could be produced with fractional charge ($<1e$) or multiple charge ($>1e$).

%Another BSM theory that includes HSCP is Universal Extra Dimensions (UED). In UED SM fields, quarks, and gluons propagate through new dimensions with the
%known SM fields representing the ground state mode in the new dimensions. States in excited modes in the new dimension would be seen as new particles.
%A new quantum number is introduced in UED for momentum conservation in the extra dimensions that forces the lightest excited particle to be stable,
%making it a dark matter. The excited states of light SM particles could have long enough lifetimes to be experimentally detectable.

