\section{Preselection \label{sec:preselection}}
Candidates for the \muononly\ analysis are tracks reconstructed in the muon system. Candidates for the \tktof\ and \multi\ analyses are tracks found in both the muon
system and the inner tracker. The \tkonly\
%and fract\ analyses require  
analysis requires only that the tracks be found in the inner tracker.
Various requirements are applied to the candidate in order to reduce tracks from background process while maintaining good efficiency for HSCP.
%For all plots shown in this section the first and last bins contain the underflow and overflow, respectively.

\subsection{Preselection for \muononly\ \label{sec:muonlypreselection}}

The \muononly\ analysis requires the candidates to have $p_T > 80$~GeV, $|\eta| < 2.1$, and valid DT or CSC hits in at least two muon stations
to reinforce the requirements applied at trigger level. 
The distributions of $\eta$ and number of muon stations are shown in Fig.~\ref{fig:MuOnlyPreselA} for data, the cosmic-ray muon control sample, and signal MC.
The discontinuities in the $\eta$ distribution are due to interfaces between the detector elements in the muon system.

Quality cuts on the \invbeta\ measurement are applied. The measurement must have at least eight degrees of freedom and the uncertainty must be less than 0.07.
Additionally the candidate must have \invbeta\ greater than one.
A potential background source is muons coming from bunch crossing windows in adjacent beam crossings. 
Candidates are required to have a measured time at vertex not be within 5ns of an expected adjacent collision.
Figure~\ref{fig:MuOnlyPreselB} shows the distribution of these quantities for data, cosmic-ray muon control sample, and signal MC.

Additional cuts are used to control the background from cosmic-ray muons. The displacement of the track
with respect to the beam spot is required to be less than 15cm in both the longitudinal and transverse direction relative to the beam line. 
The candidate $|\phi|$ must not be within 1.2--1.9 radians, this region represents tracks pointing in the vertical
direction, as is expected of cosmic-ray muons. Cosmic-ray muons travel 
through the top and bottom halves of the detector leaving hits in the muon system opposite of the candidate.
Thus, it is required that  there be no muon segments with $\eta$ within 0.1 of $-\eta_{candidate}$. Only segments separated from the candidate by at least 0.5 in $\phi$
are used to prevent candidates in the central portion of the detector to match to their own segments.
Figure~\ref{fig:MuOnlyPreselC} shows the distribution of these quantities.

\begin{figure}
\centering
  \includegraphics[clip=true, trim=0.0cm 0cm 2.8cm 0cm, width=0.44\textwidth]{figures/muonly/Selection_Comp_8TeV_Cosmic_Eta_BS}
  \includegraphics[clip=true, trim=0.0cm 0cm 2.8cm 0cm, width=0.44\textwidth]{figures/muonly/Selection_Comp_8TeV_Cosmic_MatchedStations_BS} \\
\caption[Distribution of $\eta$ and number of matched muon stations for data, cosmic-ray muon control sample, and signal MC in the \muononly\ analysis.]
{Distribution of $\eta$ (left) and number of matched muon stations (right) for data, cosmic-ray muon control sample, and signal MC in the \muononly\ analysis.}
    \label{fig:MuOnlyPreselA}
\end{figure}

\begin{figure}
\centering
  \includegraphics[clip=true, trim=0.0cm 0cm 2.8cm 0cm, width=0.44\textwidth]{figures/muonly/Selection_Comp_8TeV_Cosmic_nDof_BS}
  \includegraphics[clip=true, trim=0.0cm 0cm 2.8cm 0cm, width=0.44\textwidth]{figures/muonly/Selection_Comp_8TeV_Cosmic_TOFError_BS} \\
  \includegraphics[clip=true, trim=0.0cm 0cm 2.8cm 0cm, width=0.44\textwidth]{figures/muonly/Selection_Comp_8TeV_Cosmic_TimeAtIP_BS} \\
\caption[Distribution of number of degrees of freedom (left) and uncertainty (right) on the \invbeta\ measurement and time at vertex in 
the \muononly\ analysis for data, cosmic-ray muon control sample, and signal MC.]
{Distribution of various prelection variables in the \muononly\ analysis for data, cosmic-ray muon control sample, and signal MC.
%Top row: Disitribution of number of matched stations (left) and time at vertex (right).
Top row: Number of degrees of freedom (left) and uncertainty (right) on the \invbeta\ measurement.
Bottom row: Time at vertex.}
    \label{fig:MuOnlyPreselB}
\end{figure}

\begin{figure}
\centering
  \includegraphics[clip=true, trim=0.0cm 0cm 2.8cm 0cm, width=0.44\textwidth]{figures/muonly/Selection_Comp_8TeV_Cosmic_Dxy_BS}
  \includegraphics[clip=true, trim=0.0cm 0cm 2.8cm 0cm, width=0.44\textwidth]{figures/muonly/Selection_Comp_8TeV_Cosmic_Dz_BS} \\
  \includegraphics[clip=true, trim=0.0cm 0cm 2.8cm 0cm, width=0.44\textwidth]{figures/muonly/Selection_Comp_8TeV_Cosmic_Phi_BS}
  \includegraphics[clip=true, trim=0.0cm 0cm 2.8cm 0cm, width=0.44\textwidth]{figures/muonly/Selection_Comp_8TeV_Cosmic_SegMinEtaSep_BS}
  \caption[Distribution of transverse and longitudinal displacement, $\phi$, and $\eta$ separation to muon segments
in the \muononly\ analysis for data, cosmic-ray muon control sample, and signal MC.]
{Distribution of various prelection variables in the \muononly\ analysis for data, cosmic-ray muon control sample, and signal MC.
Top row: Disitribution of transverse (left) and longitudinal displacement (right).
Bottom row: Distribution of the $\phi$ of the candidate (left) and the $\eta$ separation of the candidate to muon segments (right). The rightmost bin in the
$\eta$ separation plot includes candidates where no segments were found.}
    \label{fig:MuOnlyPreselC}
\end{figure}

\subsection{Preselection for \tktof\ \label{sec:tktofpreselection}}

The \tktof\ analysis applies cuts on the inner tracker track, which has a much better $p_T$ and impact parameter resolution than the muon system track.
The candidate is required to have $p_T > 45$~GeV and  $|\eta| < 2.1$ to match the trigger level requirements. 
Quality cuts are applied as low quality background tracks can have mismeasured momentum and potentially high fluctuations in \dedx.
The inner track is required to have at least eight hits in the inner tracker with at least two coming from the pixel detector. At least 80\% of the hits associated with the track
must be considered valid. A cleaning procedure is applied to the hits before calculating \dedx\ and there must be at least six measurements passing this cleaning.
Figure~\ref{fig:TkMuPreselA} shows these variables for data and signal MC.

\begin{figure}
\centering
  \includegraphics[clip=true, trim=0.0cm 0cm 2.8cm 0cm, width=0.44\textwidth]{figures/tkmu/Selection_Comp_8TeV_GMStau_NOH_BS}
  \includegraphics[clip=true, trim=0.0cm 0cm 2.8cm 0cm, width=0.44\textwidth]{figures/tkmu/Selection_Comp_8TeV_GMStau_NOPH_BS} \\
  \includegraphics[clip=true, trim=0.0cm 0cm 2.8cm 0cm, width=0.44\textwidth]{figures/tkmu/Selection_Comp_8TeV_GMStau_NOHFraction_BS}
  \includegraphics[clip=true, trim=0.0cm 0cm 2.8cm 0cm, width=0.44\textwidth]{figures/tkmu/Selection_Comp_8TeV_GMStau_NOM_BS}
  \caption[Distribution of number of tracker and pixel hits, fraction of valid tracker hits, and number of \dedx\ measurements in the \tktof\ analysis for data and signal MC.]
{Distribution of various prelection variables in the \tktof\ analysis for data and signal MC.
Top row: Number of tracker (left) and pixel (right) hits.
Bottom row: Fraction of valid tracker hits (left) and number of measurements used for the \dedx\ calculation (right).}
    \label{fig:TkMuPreselA}
\end{figure}

The relative uncertainty on the candidate $p_T$ ($\sigma_{p_T}/p_T$) must be less than 0.25 and the $\chi^2$ per degree of freedom must be less than five.
While cosmic-ray muons are expected to be a negligible background in the \tktof\ analysis loose cuts are placed on the impact parameter of the track, 
these cuts are nearly 100\% efficient for signal particles.
The displacement of the track with respect to the primary vertex
with the smallest longitudinal displament must be less than 0.5~cm in both the transverse and longitudinal directions.
Figure~\ref{fig:TkMuPreselB} shows $p_T$ uncertainty, $\chi^2$ per degree of freedom, and the $d_z$ and $d_{xy}$ displacement for data and signal MC.

\begin{figure}
\centering
  \includegraphics[clip=true, trim=0.0cm 0cm 2.8cm 0cm, width=0.44\textwidth]{figures/tkmu/Selection_Comp_8TeV_GMStau_Pterr_BS}
  \includegraphics[clip=true, trim=0.0cm 0cm 2.8cm 0cm, width=0.44\textwidth]{figures/tkmu/Selection_Comp_8TeV_GMStau_Chi2_BS} \\
  \includegraphics[clip=true, trim=0.0cm 0cm 2.8cm 0cm, width=0.44\textwidth]{figures/tkmu/Selection_Comp_8TeV_GMStau_Dxy_BS}
  \includegraphics[clip=true, trim=0.0cm 0cm 2.8cm 0cm, width=0.44\textwidth]{figures/tkmu/Selection_Comp_8TeV_GMStau_Dz_BS}
  \caption[Distribution of relative \pt\ uncertainty, $\chi^2$ per degree of freedom, and transverse and longitudinal
displacement in the \tktof\ analysis for data and signal MC.]
{Distribution of various prelection variables in the \tktof\ analysis for data and signal MC.
Top row: Relative $p_T$ uncertainty (left) and $\chi^2$ per degree of freedom (right).
Bottom row: Displacement in the transverse (left) and longitudinal (right) directions.}
    \label{fig:TkMuPreselB}
\end{figure}

Isolation cuts are also applied in the \tktof\ analysis. Isolated means that there not be many high energy particles near the candidate.
This is required to reduce the background from jets where overlapping tracks could give
anomolously high \dedx\ values. The isolation cuts are kept very loose as the goal is not to find isolated particles but just to reject very high energy jets.
The sum of the momentum of the tracks within 0.3 in $\eta-\phi$ space of the candidate (excluding the candidate itself) is required to be less than 50 GeV. Additionally the total
amount of energy measured in the calorimeter within a radius of 0.3 in $\eta-\phi$ space to the candidate divided by the candidate momentum must be less than 0.3.

Additionally, the \tktof\ analysis uses the same cuts on the \invbeta\ uncertainty and number of measurements as the \muononly\ analysis.
Figure~\ref{fig:TkMuPreselC} shows the isolation and \invbeta\ variables for data and signal MC.

\begin{figure}
\centering
  \includegraphics[clip=true, trim=0.0cm 0cm 2.8cm 0cm, width=0.44\textwidth]{figures/tkmu/Selection_Comp_8TeV_GMStau_IsolT_BS}
  \includegraphics[clip=true, trim=0.0cm 0cm 2.8cm 0cm, width=0.44\textwidth]{figures/tkmu/Selection_Comp_8TeV_GMStau_IsolE_BS} \\
  \includegraphics[clip=true, trim=0.0cm 0cm 2.8cm 0cm, width=0.44\textwidth]{figures/tkmu/Selection_Comp_8TeV_GMStau_TOFError_BS}
  \includegraphics[clip=true, trim=0.0cm 0cm 2.8cm 0cm, width=0.44\textwidth]{figures/tkmu/Selection_Comp_8TeV_GMStau_nDof_BS}
  \caption[Distribution of tracker and calorimeter isolation as well as the \invbeta\ measurement number of degrees of freedom and uncertainty
in the \tktof\ analysis for data and signal MC.]
{Distribution of various prelection variables in the \tktof\ analysis for data and signal MC.
Top row: Sum momentum of tracks within 0.3 (left) and calorimeter energy within 0.3 divided by track momentum (right).
Bottom row: Distribution of the \invbeta\ measurement uncertainty (left) and the number of degrees of freedom (right).}
    \label{fig:TkMuPreselC}
\end{figure}

\subsection{Preselection for \tkonly\ and \multi\ \label{sec:otherpreselection}}

The \tkonly\ analysis applies the same preselection as the \tktof\ analysis except the cuts on the timing measurement are not applied as the candidates
are not required to be reconstructed in the muon system. 
%The \fract\ analysis uses preselection like \tkonly\ but inverting the \ih\ requirement
%to be less than 2.8 and requiring no tracks with \pt\ greater than 45 GeV to have an opening angle with the candidate greater than 2.8 radians.

The \multi\ analysis applies
the same selection criteria as the \tktof\ analysis except the cut on relative isolation less than 0.3 and the cleaning of the hits used for the \dedx\ calculation
is not done. The cleaning procedure is not applied because the amount of charge deposited is proportional to $Q^2$ meaning that even a $Q=2e$ HSCP will
deposit four times as much charge as a $Q=1e$ HSCP. As the tracker saturates for a charge approximately three times that expected for a MIP many of the hits from $Q>1e$ HSCP
will be saturated and this can confuse the cleaning procedure. Additionally, as the high charge samples deposit so much charge, 
there will still be good signal/background separation even with longer tails in the \dedx\ distribution.
Figure~\ref{fig:Multi} shows the number of measurements passing the cleaning for multiply charged samples.
% and the opening angle described above for fractionally charged samples.

\begin{figure}
\centering
%  \includegraphics[clip=true, trim=0.0cm 0cm 2.8cm 0cm, width=0.44\textwidth]{figures/fract/Selection_Comp_8TeV_DY_OpenAngle_BS}
  \includegraphics[clip=true, trim=0.0cm 0cm 2.8cm 0cm, width=0.44\textwidth]{figures/multi/Selection_Comp_8TeV_DY_QG_NOM_BS}
  \caption{Distribution of number of \dedx\ measurements passing cleaning for samples of three different charges
    \label{fig:Multi}}
\end{figure}

\subsection{Summary of Preselection \label{sec:summarypreselection}}

The preselection criteria applied on the track in the muon system used in the \muononly, \tktof, and \multi\ analyses are summarized in Table~\ref{tab:preselectionSA}.
The preselection criteria applied on the track in the inner tracker used in the \tktof, \tkonly, and \multi\ analyses are summarized in Table~\ref{tab:preselectionTk}.

%\begin{table}
% \begin{center}
%  \caption{Preselection criteria used in the various analyses}
%     \label{tab:preselection}
%  \begin{tabular}{|l|c|c|c|c|} \hline
%                                            & Muon               & Muon+ & {\em multiple}            & Track       \\
%                                            & Only               & track & {\em charge}              & Only        \\ \hline
%   Track Type                               & Muon               & \multicolumn{2}{c|}{Inner + muon} & Inner       \\ \hline
%   $|\eta|$                                 & \multicolumn{4}{c|}{$< 2.1$}                                         \\ \hline
%   $p_T$ (GeV)                              & $> 80$             & \multicolumn{3}{c|}{$> 45$}                     \\ \hline
%   $d_z$ and $d_{xy}$ (cm)                  & $< 15$             & \multicolumn{3}{c|}{$< 0.5$}                    \\ \hline
%   \# DT or CSC Stations                    & $> 1$              & \multicolumn{3}{c|}{/}                          \\ \hline
%   Opp. segment $|\eta|$ difference         & $> 0.1$            & \multicolumn{3}{c|}{/}                          \\ \hline
%   $|\phi|$                                 & $< 1.2$ OR $> 1.9$ & \multicolumn{3}{c|}{/}                          \\ \hline
%   $|\delta t|$ to other beam crossing (ns) & $> 5$              & \multicolumn{3}{c|}{/}                          \\ \hline
%   \# TOF Measurements                      & \multicolumn{3}{c|}{$> 7$}                             & /           \\ \hline
%   $\sigma_{1/\beta}$                       & \multicolumn{3}{c|}{$< 0.07$}                          & /           \\ \hline
%   $1/\beta$                                & \multicolumn{3}{c|}{$> 1$}                             & /           \\ \hline
%   $\sigma_{p_T}/p_T$                       & /                  & \multicolumn{3}{c|}{$< 0.25$}                   \\ \hline
%   Track $\chi^2/d.o.f$                     & /                  & \multicolumn{3}{c|}{$< 5$}                      \\ \hline
%   \# Pixel Hits                            & /                  & \multicolumn{3}{c|}{$> 1$}                      \\ \hline
%   \# Tracker Hits                          & /                  & \multicolumn{3}{c|}{$> 7$}                      \\ \hline
%   Frac. Valid Hits                         & /                  & \multicolumn{3}{c|}{$> 0.8$}                    \\ \hline
%   \# \dedx\ Measurements                   & /                  & \multicolumn{3}{c|}{$> 5$}                      \\ \hline
%   \ih\ (MeV/cm)                            & /                  & \multicolumn{3}{c|}{ $> 3.0$}                   \\ \hline
%   \dedx\ Strip Shape Test                  & /                  & yes     & no                        & yes       \\ \hline
%   $\Sigma p_T^{trk} (\Delta R < 0.3)$ (GeV)& /                  & \multicolumn{3}{c|}{$< 50$}                     \\ \hline
%   $E_{cal}(\Delta R < 0.3)/p$              & /                  & $< 0.3$ & /                         & $< 0.3$   \\ \hline
%  \end{tabular}
% \end{center}
%\end{table}

\begin{table}
 \begin{center}
  \caption{Preselection criteria on the muon system track used in the various analyses as defined in the text.
     \label{tab:preselectionSA}}
  \begin{tabular}{|l|c|c|c|} \hline
                                            & \muononly\ & \tktof\  &  $|Q|>1e$    \\ \hline
   \# TOF measurements                      & \multicolumn{3}{c|}{$> 7$}   \\ \hline
   $\sigma_{1/\beta}$                       & \multicolumn{3}{c|}{$< 0.07$}\\ \hline
   $1/\beta$                                & \multicolumn{3}{c|}{$> 1$}   \\ \hline
   $|\eta|$                                 & $< 2.1$              & \multicolumn{2}{c|}{$-$} \\ \hline
   $p_T$ ($GeV/c$)                            & $> 80$      & \multicolumn{2}{c|}{$-$} \\ \hline
   $d_z$ and $d_{xy}$ (cm)                  & $< 15$      & \multicolumn{2}{c|}{$-$} \\ \hline
   \# DT or CSC stations                         & $> 1$      & \multicolumn{2}{c|}{$-$} \\ \hline
   Opp. segment $|\eta|$ difference              & $> 0.1$    & \multicolumn{2}{c|}{$-$} \\ \hline
   $|\phi|$                                      & $< 1.2$ OR $> 1.9$    & \multicolumn{2}{c|}{$-$} \\ \hline
   $|\delta t|$ to other beam crossing (ns)      & $>5$    & \multicolumn{2}{c|}{$-$} \\ \hline
  \end{tabular}
 \end{center}
\end{table}

\begin{table}
 \begin{center}
  \caption{Preselection criteria on the inner tracker track used in the various analyses as defined in the text.
     \label{tab:preselectionTk}}
  \begin{tabular}{|l|c|c|c|} \hline
                                            & \tktof\ & \tkonly\  &  $|Q|>1e$    \\ \hline
   $|\eta|$                                 & \multicolumn{3}{c|}{$< 2.1$}                            \\ \hline
   $p_T$ ($GeV/c$)                            & \multicolumn{3}{c|}{$> 45$}               \\ \hline
   $d_z$ and $d_{xy}$ (cm)                  & \multicolumn{3}{c|}{$< 0.5$}              \\ \hline
   $\sigma_{p_T}/p_T$                       & \multicolumn{3}{c|}{$< 0.25$}             \\ \hline
   Track $\chi^2/{n_d}$           & \multicolumn{3}{c|}{$< 5$}                    \\ \hline
   \# Pixel hits                            & \multicolumn{3}{c|}{$> 1$}                \\ \hline
   \# Tracker hits                          & \multicolumn{3}{c|}{$> 7$}                \\ \hline
   Frac. Valid hits                         & \multicolumn{3}{c|}{$> 0.8$}              \\ \hline
   $\Sigma p_{T}^{trk} (\Delta R < 0.3)$ ($GeV/c$) & \multicolumn{3}{c|}{$< 50$}             \\ \hline
   \# \dedx\ measurements                   & \multicolumn{3}{c|}{$> 5$}                \\ \hline
   \dedx\ strip shape test                  & \multicolumn{2}{c|}{yes}       & no       \\ \hline
   $E_{cal}(\Delta R < 0.3)/p$              & \multicolumn{2}{c|}{$< 0.3$}   & $-$      \\ \hline
  \end{tabular}
 \end{center}
\end{table}

The total preselection efficiency is shown in Tables~\ref{tab:preselectionEff} and ~\ref{tab:preselectionEffA} for the SUSY and modified DY samples, respectively.
The efficiencies are presented with respect to HSCP reconstructed as a track in CMS.
The inefficiency for the \muononly\ analysis mostly arises from the cuts used to suppress the background from cosmic-ray muons.
The other analyses lose efficiency due to quality requirements on the inner track and \dedx\ measurement
which are necessary to constrain the background from misreconstructed tracks.
Additionally, as CMS reconstruction generally assumes signatures of SM particles, HSCP tracks can have a lower quality than a SM particle would.
The requirements are set trying to balance keeping the signal efficiency high while maintaining a low background contamination of the signal region.


\begin{table}
 \begin{center}
  \caption[Preselection efficiency for a few benchmark SUSY samples in the \muononly, \tktof, and \tkonly\ analyses]
{Preselection efficiency for a few benchmark SUSY samples in each analysis.  
This efficiency is with respect to the reconstructed HSCP candidate (i.e. muon system track for the \muononly\ analysis and muon system plus inner tracker 
for the \tktof\ analysis).
The fraction of glueballs assumed for the gluino samples is given in parantheses at the end of the signal name.}
     \label{tab:preselectionEff}
   \begin{tabular}{|l|c|c|c|} \hline
Model         & \muononly\        & \tktof\        & \tkonly\  \\ \hline
%         & muon-           & track        & track  \\
%Model                    & only           & +muon        & only   \\ \hline
%Gluino 500    & \multirow{2}{*}{44\%} & \multirow{2}{*}{-}    & \multirow{2}{*}{-}  \\
%GeV (1.0)     &                         &                       &                  \\ \hline
%Guino 1000    & \multirow{2}{*}{40\%} & \multirow{2}{*}{-}    & \multirow{2}{*}{-} \\
%GeV (1.0)     &                         &                       &                  \\ \hline
%Gluino 500    & \multirow{2}{*}{44\%} & \multirow{2}{*}{60\%}    & \multirow{2}{*}{70\%} \\
%GeV (0.1)     &                         &                       &                     \\ \hline
%Gluino 1000   & \multirow{2}{*}{43\%} & \multirow{2}{*}{42\%} & \multirow{2}{*}{51\%} \\
%GeV (0.1)     &                         &                       &                     \\ \hline
%Gluino(CS)    & \multirow{2}{*}{-}    & \multirow{2}{*}{-}    & \multirow{2}{*}{64\%} \\
%5500 GeV (0.1) &                       &                       &                       \\ \hline
%Gluino(CS)    & \multirow{2}{*}{-}    & \multirow{2}{*}{-}    & \multirow{2}{*}{47\%} \\
%1000 GeV(0.1) &                         &                       &                     \\ \hline
%Stop 600      & \multirow{2}{*}{48\%} & \multirow{2}{*}{53\%} & \multirow{2}{*}{61\%} \\
%GeV           &                         &                       &                     \\ \hline
%Stop (CS)     & \multirow{2}{*}{56\%} & \multirow{2}{*}{-}    & \multirow{2}{*}{56\%} \\
%600 GeV       &                         &                       &                     \\ \hline
%GMSB Stau     & \multirow{2}{*}{-}    & \multirow{2}{*}{76\%}    & \multirow{2}{*}{78\%} \\
%370 GeV       &                         &                       &                     \\ \hline%


Gluino 500 GeV (1.0)   & 44\% & -    & -  \\
%GeV (1.0)     &     	      	      	&     	      	      	&     	      	   \\ \hline
Guino 1000 GeV (1.0)   & 40\% & -    & - \\ 
%GeV (1.0)     &     	      	      	&     	      	      	&     	      	   \\ \hline
Gluino 500 GeV (0.1)   & 44\% & 60\%    & 70\% \\
%GeV (0.1)     &     	      	      	&     	      	      	&     	      	      \\ \hline
Gluino 1000 GeV (0.1)  & 43\% & 42\% & 51\% \\
%GeV (0.1)     &     	      	      	&     	      	      	&     	      	      \\ \hline
Gluino(CS) 500 GeV (0.1)   & -    & -    & 64\% \\
%500 GeV (0.1) &                       &                       &                       \\ \hline
Gluino(CS) 1000 GeV (0.1)   & -    & -    & 47\% \\
%1000 GeV(0.1) &     	      	      	&     	      	      	&     	      	      \\ \hline
Stop 600 GeV     & 48\% & 53\% & 61\% \\
%GeV           &     	      	      	&     	      	      	&     	      	      \\ \hline
Stop (CS) GeV    & 56\% & -    & 56\% \\
%600 GeV       &     	      	      	&     	      	      	&     	      	      \\ \hline
GMSB Stau 370 GeV    & -    & 76\%    & 78\% \\
%370 GeV       &     	      	      	&     	      	      	&     	      	      \\ \hline
\hline
   \end{tabular}
 \end{center}
\end{table}


\begin{table}
 \begin{center}
  \caption
[Preselection efficiency for a few benchmark modified DY samples in the \tktof, \tkonly, and \multi\ analyses.]
{Preselection efficiency for a few benchmark modified DY samples in each analysis.
This efficiency is with respect to the reconstructed HSCP candidate (i.e. muon system plus inner track for the \multi\ analysis and inner track for the \tkonly\ analysis).}
     \label{tab:preselectionEffA}
   \begin{tabular}{|l|c|c|c|} \hline
%              & track        & track        & {\em multiple} \\
%Model         & +muon        & only         & {\em charge}          \\ \hline
%%DY Q2o3       & \multirow{2}{*}{15\%}    & \multirow{2}{*}{17\%} & \multirow{2}{*}{-} \\
%%400 GeV       &                          &                       &                    \\ \hline
%DY $Q = 1e$         & \multirow{2}{*}{72\%}    & \multirow{2}{*}{76\%} & \multirow{2}{*}{75\%} \\
%600 GeV       &                          &                       &                       \\ \hline
%DY $Q = 3e$         & \multirow{2}{*}{-}    & \multirow{2}{*}{-}    & \multirow{2}{*}{71\%} \\
%600 GeV       &                          &                       &                       \\ \hline
%DY $Q = 5e$         & \multirow{2}{*}{-}    & \multirow{2}{*}{-}    & \multirow{2}{*}{50\%} \\
%600 GeV       &                          &                       &                       \\ \hline
%DY $Q 7e$         & \multirow{2}{*}{-}       & \multirow{2}{*}{-}    & \multirow{2}{*}{37\%} \\
%600 GeV       &                          &                       &                       \\ \hline

Model     & \tktof\    & \tkonly\        & \multi\ \\ \hline
DY $Q = 1e$ 600 GeV        & 72\%    & 76\% & 75\% \\
DY $Q = 3e$ 600 GeV        & -    & -    & 71\% \\
DY $Q = 5e$ 600 GeV        & -    & -    & 50\% \\
DY $Q = 7e$ 600 GeV        & -       & -    & 37\% \\
\hline
   \end{tabular}
 \end{center}
\end{table}


The distributions of $p_T$ and \invbeta\ for the \muononly\ analysis for data, cosmic-ray muon control sample, and various signal models is 
shown in Figure~\ref{fig:MuOnlySelVar} after applying the preselection requirements. Figure~\ref{fig:TkMuSelVar} shows the $p_T$, \invbeta, and \dedx\
distributions after applying the \tktof\ preselection cuts for data and various signal models.

\begin{figure}
\centering
  \includegraphics[clip=true, trim=0.0cm 0cm 2.8cm 0cm, width=0.44\textwidth]{figures/muonly/Selection_Comp_8TeV_Cosmic_Pt_BS}
  \includegraphics[clip=true, trim=0.0cm 0cm 2.8cm 0cm, width=0.44\textwidth]{figures/muonly/Selection_Comp_8TeV_Cosmic_TOF_BS} \\
  \caption[Distribution of \invbeta\ and \pt\ in the \muononly\ analysis for data, cosmic-ray muon control sample, and signal MC]
{Distribution of selection variables for data, cosmic-ray muon control sample, and signal MC.
Left: Distribution of $p_T$. Right: Distribution of \invbeta.}
    \label{fig:MuOnlySelVar}
\end{figure}

\begin{figure}
\centering
  \includegraphics[clip=true, trim=0.0cm 0cm 2.8cm 0cm, width=0.44\textwidth]{figures/tkmu/Selection_Comp_8TeV_GMStau_Pt_BS}
  \includegraphics[clip=true, trim=0.0cm 0cm 2.8cm 0cm, width=0.44\textwidth]{figures/tkmu/Selection_Comp_8TeV_GMStau_TOF_BS} \\
  \includegraphics[clip=true, trim=0.0cm 0cm 2.8cm 0cm, width=0.44\textwidth]{figures/tkmu/Selection_Comp_8TeV_GMStau_Is_BS}
  \includegraphics[clip=true, trim=0.0cm 0cm 2.8cm 0cm, width=0.44\textwidth]{figures/tkmu/Selection_Comp_8TeV_GMStau_Im_BS}
  \caption[Distribution of selection variables in the \tktof\ analysis for data and signal MC.]
{Distribution of section variables for data and signal MC.
Top row: Distribution of $p_T$ (left) and \invbeta\ (right).
Bottom row: Distribution of \ias\ (left) and \ih\ (right).}
    \label{fig:TkMuSelVar}
\end{figure}

\subsection{Tag and Probe Studies \label{sec:TagProbe}}
The study of the agreement between data and MC for numerous muon qualities is done by the Muon Physics Object Group (POG) inside of CMS.
The group provides scale factors for correcting MC to match data.
For all of the analyses except for \muononly\  it is sufficient to use results obtained from this group as the muon qualities
they use are common within CMS.
However, as the \muononly\ analysis uses numerous variables which are unique to it, the results from the muon POG are not applicable.

For this reason, additional studies were performed to test the agreement of MC with data.
The efficiency of the selections was checked with a tag and probe procedure
(from the Muon POG) using muons from the decay of the Z boson. Z bosons decay to a particle and its anti-particle with the invariant mass of the particle--anti-particle
pair equal to the mass of the Z boson they were created from.

The tag and probe procedure proceeds by
requiring one muon, the tag muon, be found with a very stringent selection trying to assure that this is a good muon. The tag muon is required to pass
a tight selection recommended by the Muon POG and to match to an object that triggered the readout of CMS. 
The last requirement assures that no bias is introduced by the need for the event to readout.
Additionally, the tag must pass the requirements of a skim that was used to reduce the data size to a level
making processing reasonable. The skim requirements are at least three \dedx\ measurements and \ih\ $> 3.0$ or \ih\ $< 2.8$.

Then a set of probe candidates are defined as tracks reconstructed in the inner tracker with no requirement
of muon system activity. The probes are required to have $p_T > 40$ GeV, $|\eta| < 2.1$, and the opposite charge of the tag muon. The invariant mass of the tag and
probe is then required to be within 10 GeV of the mass of the Z boson, 91 GeV.

There are few processes other than Z boson decay that will lead to tag and probe pairs having an invariant mass around the mass of the Z boson. Thus it is likely that the
probe is a muon. The efficiency that a muon passes the preselection in the \muononly\ analysis can then be found by looking at the efficiency that the probe
passes the preselection. The efficiencies in data and MC can be compared and any possible scale factors calculated. The residual background from non Z boson decay in the
mass window is accounted for by a fit described below. The MC sample used only contains the creation of Z bosons, background processes are not included.

A simultaneous fit to pairs originating from Z bosons and pairs from background is performed using the
sum of two Voigtians to represent Z bosons and an exponential for the background.
Figures~\ref{fig:MuOnlyTagProbeFitData} and ~\ref{fig:MuOnlyTagProbeFitMC} show sample fits to data and MC, respectively.
It can be seen that the fits match well. The efficiency is extracted from these fits using a procedure from the muon POG.

\begin{figure}
 \begin{center}
  \includegraphics[width=0.95\textwidth]{figures/muonly/FitCanvasDataPtBin0}
 \end{center}
 \caption[Example fits to invariant mass distributions
in the tag and probe procedure for the \muononly\ analysis for data.]
{Example fits to invariant mass distributions
in the tag and probe procedure for the \muononly\ analysis for data.
Top left: Mass distribution and fit for probes passing the preselection. Top right: Mass distribution and fit for probes failing the preselection.
Bottom left: Mass distribution and fit for all probes. Bottom right: Parameters extracted from the fit.}
    \label{fig:MuOnlyTagProbeFitData}
\end{figure}

\begin{figure}
 \begin{center}
  \includegraphics[width=0.95\textwidth]{figures/muonly/FitCanvasMCEtaBin6}
 \end{center}
 \caption[Example fits to invariant mass distributions
in the tag and probe procedure for the \muononly\ analysis for MC.]
{Same as Fig.~\ref{fig:MuOnlyTagProbeFitData} but for MC.}
    \label{fig:MuOnlyTagProbeFitMC}
\end{figure}

Figure~\ref{fig:MuOnlyTagProbeEff} shows the efficiency for the probes
to pass the preselection, except for the selection on $p_T$, against the probe
$p_T$, $\eta$, and the number of vertices in the event.
Overall the efficiency is approximately 75\% in data and 80\% in MC.
The efficiency is mostly flat versus $p_T$ and number of vertices but does depend
on $\eta$.  MC is scaled by an $\eta$ dependent scale factor to correct for the discrepancy.

\begin{figure}
 \begin{center}
  \includegraphics[clip=true, trim=0.0cm 0cm 1.4cm 0cm, width=0.44\textwidth]{figures/muonly/pteff_Comp}
  \includegraphics[clip=true, trim=0.0cm 0cm 1.4cm 0cm, width=0.44\textwidth]{figures/muonly/etaeff_Comp}
  \includegraphics[clip=true, trim=0.0cm 0cm 1.4cm 0cm, width=0.44\textwidth]{figures/muonly/PVeff_Comp}
 \end{center}
 \caption[Efficiency to pass preselection cuts for the \muononly\ analysis versus \pt, $\eta$, and number of primary vertices]
{Efficiency to pass preselection cuts for the \muononly\ analysis.
Top row: Versus $p_T$ (left) and $\eta$ (right)
Bottom row: Versus number of primary vertices (right).}
   \label{fig:MuOnlyTagProbeEff}
\end{figure}